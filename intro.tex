
In a 1$\times$1 grid, there are no crossing spaces, so only one possible pattern can be drawn, the trivial one-unit loop.
\begin{marginfigure}[+3.5cm]
\begin{flushright}
\scalebox{0.7}{
\begin{tikzpicture} [framed,background rectangle/.style={line width=4pt, line cap=round,draw=black}]
 \draw [line width=4pt, line cap=round] (0.875,1.75) -- (0.4375,2.1875); 
 \draw [line width=4pt, line cap=round] (0.875,3.5) -- (0.4375,3.9375); 
 \draw [line width=4pt, line cap=round] (0.875,5.25) -- (0.4375,5.6875); 
 \draw [line width=4pt, line cap=round] (0.875,7) -- (0.4375,7.4375); 
 \draw [line width=4pt, line cap=round] (0.4375,8.3125) -- (0.4375,9.1875); 
 \draw [line width=4pt, line cap=round] (0.875,10.5) -- (0.4375,10.9375); 
 \draw [line width=4pt, line cap=round] (0.875,12.25) -- (0.4375,12.6875); 
 \draw [line width=4pt, line cap=round] (0.875,14) -- (0.4375,14.4375); 
 \draw [line width=4pt, line cap=round] (0.875,15.75) -- (0.4375,16.1875); 
 \draw [line width=4pt, line cap=round] (0.875,17.5) -- (0.4375,17.9375); 
 \draw [line width=4pt, line cap=round] (0.4375,18.8125) -- (0.4375,19.6875); 
 \draw [line width=4pt, line cap=round] (0.875,1.75) -- (1.3125,1.3125); 
 \draw [line width=4pt, line cap=round] (0.4375,1.3125) -- (0.4375,0.4375); 
 \draw [line width=4pt, line cap=round] (0.4375,1.3125) -- (0.65625,1.53125); 
 \draw [line width=4pt, line cap=round] (0.4375,0.4375) -- (1.3125,0.4375); 
 \draw [line width=4pt, line cap=round] (1.75,2.625) -- (1.3125,2.1875); 
 \draw [line width=4pt, line cap=round] (1.3125,2.1875) -- (1.09375,1.96875); 
 \draw [line width=4pt, line cap=round] (0.875,3.5) -- (1.3125,3.0625); 
 \draw [line width=4pt, line cap=round] (1.3125,3.0625) -- (1.53125,2.84375); 
 \draw [line width=4pt, line cap=round] (0.4375,3.0625) -- (0.4375,2.1875); 
 \draw [line width=4pt, line cap=round] (0.4375,3.0625) -- (0.65625,3.28125); 
 \draw [line width=4pt, line cap=round] (0.875,1.75) -- (0.4375,2.1875); 
 \draw [line width=4pt, line cap=round] (0.875,5.25) -- (1.3125,4.8125); 
 \draw [line width=4pt, line cap=round] (0.4375,4.8125) -- (0.4375,3.9375); 
 \draw [line width=4pt, line cap=round] (0.4375,4.8125) -- (0.65625,5.03125); 
 \draw [line width=4pt, line cap=round] (0.875,3.5) -- (0.4375,3.9375); 
 \draw [line width=4pt, line cap=round] (1.75,6.125) -- (1.3125,5.6875); 
 \draw [line width=4pt, line cap=round] (1.3125,5.6875) -- (1.09375,5.46875); 
 \draw [line width=4pt, line cap=round] (0.875,7) -- (1.3125,6.5625); 
 \draw [line width=4pt, line cap=round] (1.3125,6.5625) -- (1.53125,6.34375); 
 \draw [line width=4pt, line cap=round] (0.4375,6.5625) -- (0.4375,5.6875); 
 \draw [line width=4pt, line cap=round] (0.4375,6.5625) -- (0.65625,6.78125); 
 \draw [line width=4pt, line cap=round] (0.875,5.25) -- (0.4375,5.6875); 
 \draw [line width=4pt, line cap=round] (1.75,7.875) -- (1.3125,7.4375); 
 \draw [line width=4pt, line cap=round] (1.3125,7.4375) -- (1.09375,7.21875); 
 \draw [line width=4pt, line cap=round] (0.4375,8.3125) -- (0.4375,7.4375); 
 \draw [line width=4pt, line cap=round] (0.875,7) -- (0.4375,7.4375); 
 \draw [line width=4pt, line cap=round] (1.75,9.625) -- (1.3125,9.1875); 
 \draw [line width=4pt, line cap=round] (0.875,10.5) -- (1.3125,10.0625); 
 \draw [line width=4pt, line cap=round] (1.3125,10.0625) -- (1.53125,9.84375); 
 \draw [line width=4pt, line cap=round] (0.4375,10.0625) -- (0.4375,9.1875); 
 \draw [line width=4pt, line cap=round] (0.4375,10.0625) -- (0.65625,10.28125); 
 \draw [line width=4pt, line cap=round] (1.75,11.375) -- (1.3125,10.9375); 
 \draw [line width=4pt, line cap=round] (1.3125,10.9375) -- (1.09375,10.71875); 
 \draw [line width=4pt, line cap=round] (0.875,12.25) -- (1.3125,11.8125); 
 \draw [line width=4pt, line cap=round] (1.3125,11.8125) -- (1.53125,11.59375); 
 \draw [line width=4pt, line cap=round] (0.4375,11.8125) -- (0.4375,10.9375); 
 \draw [line width=4pt, line cap=round] (0.4375,11.8125) -- (0.65625,12.03125); 
 \draw [line width=4pt, line cap=round] (0.875,10.5) -- (0.4375,10.9375); 
 \draw [line width=4pt, line cap=round] (1.75,13.125) -- (1.3125,12.6875); 
 \draw [line width=4pt, line cap=round] (1.3125,12.6875) -- (1.09375,12.46875); 
 \draw [line width=4pt, line cap=round] (0.875,14) -- (1.3125,13.5625); 
 \draw [line width=4pt, line cap=round] (1.3125,13.5625) -- (1.53125,13.34375); 
 \draw [line width=4pt, line cap=round] (0.4375,13.5625) -- (0.4375,12.6875); 
 \draw [line width=4pt, line cap=round] (0.4375,13.5625) -- (0.65625,13.78125); 
 \draw [line width=4pt, line cap=round] (0.875,12.25) -- (0.4375,12.6875); 
 \draw [line width=4pt, line cap=round] (1.75,14.875) -- (1.3125,14.4375); 
 \draw [line width=4pt, line cap=round] (1.3125,14.4375) -- (1.09375,14.21875); 
 \draw [line width=4pt, line cap=round] (0.875,15.75) -- (1.3125,15.3125); 
 \draw [line width=4pt, line cap=round] (1.3125,15.3125) -- (1.53125,15.09375); 
 \draw [line width=4pt, line cap=round] (0.4375,15.3125) -- (0.4375,14.4375); 
 \draw [line width=4pt, line cap=round] (0.4375,15.3125) -- (0.65625,15.53125); 
 \draw [line width=4pt, line cap=round] (0.875,14) -- (0.4375,14.4375); 
 \draw [line width=4pt, line cap=round] (0.875,17.5) -- (1.3125,17.0625); 
 \draw [line width=4pt, line cap=round] (0.4375,17.0625) -- (0.4375,16.1875); 
 \draw [line width=4pt, line cap=round] (0.4375,17.0625) -- (0.65625,17.28125); 
 \draw [line width=4pt, line cap=round] (0.875,15.75) -- (0.4375,16.1875); 
 \draw [line width=4pt, line cap=round] (1.75,18.375) -- (1.3125,17.9375); 
 \draw [line width=4pt, line cap=round] (1.3125,17.9375) -- (1.09375,17.71875); 
 \draw [line width=4pt, line cap=round] (0.4375,18.8125) -- (0.4375,17.9375); 
 \draw [line width=4pt, line cap=round] (0.875,17.5) -- (0.4375,17.9375); 
 \draw [line width=4pt, line cap=round] (1.3125,20.5625) -- (0.4375,20.5625); 
 \draw [line width=4pt, line cap=round] (0.4375,20.5625) -- (0.4375,19.6875); 
 \draw [line width=4pt, line cap=round] (2.1875,0.4375) -- (1.3125,0.4375); 
 \draw [line width=4pt, line cap=round] (2.1875,1.3125) -- (2.1875,2.1875); 
 \draw [line width=4pt, line cap=round] (2.1875,2.1875) -- (1.96875,2.40625); 
 \draw [line width=4pt, line cap=round] (1.75,2.625) -- (1.3125,2.1875); 
 \draw [line width=4pt, line cap=round] (0.875,1.75) -- (1.3125,1.3125); 
 \draw [line width=4pt, line cap=round] (1.3125,1.3125) -- (2.1875,1.3125); 
 \draw [line width=4pt, line cap=round] (2.625,3.5) -- (2.1875,3.9375); 
 \draw [line width=4pt, line cap=round] (2.1875,3.9375) -- (1.3125,3.9375); 
 \draw [line width=4pt, line cap=round] (1.3125,3.9375) -- (1.09375,3.71875); 
 \draw [line width=4pt, line cap=round] (0.875,3.5) -- (1.3125,3.0625); 
 \draw [line width=4pt, line cap=round] (1.75,2.625) -- (2.1875,3.0625); 
 \draw [line width=4pt, line cap=round] (2.625,5.25) -- (2.1875,5.6875); 
 \draw [line width=4pt, line cap=round] (1.75,6.125) -- (1.3125,5.6875); 
 \draw [line width=4pt, line cap=round] (0.875,5.25) -- (1.3125,4.8125); 
 \draw [line width=4pt, line cap=round] (1.3125,4.8125) -- (2.1875,4.8125); 
 \draw [line width=4pt, line cap=round] (2.1875,4.8125) -- (2.40625,5.03125); 
 \draw [line width=4pt, line cap=round] (2.625,7) -- (2.1875,7.4375); 
 \draw [line width=4pt, line cap=round] (1.75,7.875) -- (1.3125,7.4375); 
 \draw [line width=4pt, line cap=round] (0.875,7) -- (1.3125,6.5625); 
 \draw [line width=4pt, line cap=round] (1.75,6.125) -- (2.1875,6.5625); 
 \draw [line width=4pt, line cap=round] (2.625,8.75) -- (2.1875,9.1875); 
 \draw [line width=4pt, line cap=round] (1.75,9.625) -- (1.3125,9.1875); 
 \draw [line width=4pt, line cap=round] (1.3125,9.1875) -- (1.3125,8.3125); 
 \draw [line width=4pt, line cap=round] (1.3125,8.3125) -- (1.53125,8.09375); 
 \draw [line width=4pt, line cap=round] (1.75,7.875) -- (2.1875,8.3125); 
 \draw [line width=4pt, line cap=round] (2.625,10.5) -- (2.1875,10.9375); 
 \draw [line width=4pt, line cap=round] (1.75,11.375) -- (1.3125,10.9375); 
 \draw [line width=4pt, line cap=round] (0.875,10.5) -- (1.3125,10.0625); 
 \draw [line width=4pt, line cap=round] (1.75,9.625) -- (2.1875,10.0625); 
 \draw [line width=4pt, line cap=round] (2.1875,11.8125) -- (2.1875,12.6875); 
 \draw [line width=4pt, line cap=round] (2.1875,12.6875) -- (1.96875,12.90625); 
 \draw [line width=4pt, line cap=round] (1.75,13.125) -- (1.3125,12.6875); 
 \draw [line width=4pt, line cap=round] (0.875,12.25) -- (1.3125,11.8125); 
 \draw [line width=4pt, line cap=round] (1.75,11.375) -- (2.1875,11.8125); 
 \draw [line width=4pt, line cap=round] (2.625,14) -- (2.1875,14.4375); 
 \draw [line width=4pt, line cap=round] (1.75,14.875) -- (1.3125,14.4375); 
 \draw [line width=4pt, line cap=round] (0.875,14) -- (1.3125,13.5625); 
 \draw [line width=4pt, line cap=round] (1.75,13.125) -- (2.1875,13.5625); 
 \draw [line width=4pt, line cap=round] (2.625,15.75) -- (2.1875,16.1875); 
 \draw [line width=4pt, line cap=round] (2.1875,16.1875) -- (1.3125,16.1875); 
 \draw [line width=4pt, line cap=round] (1.3125,16.1875) -- (1.09375,15.96875); 
 \draw [line width=4pt, line cap=round] (0.875,15.75) -- (1.3125,15.3125); 
 \draw [line width=4pt, line cap=round] (1.75,14.875) -- (2.1875,15.3125); 
 \draw [line width=4pt, line cap=round] (2.625,17.5) -- (2.1875,17.9375); 
 \draw [line width=4pt, line cap=round] (1.75,18.375) -- (1.3125,17.9375); 
 \draw [line width=4pt, line cap=round] (0.875,17.5) -- (1.3125,17.0625); 
 \draw [line width=4pt, line cap=round] (1.3125,17.0625) -- (2.1875,17.0625); 
 \draw [line width=4pt, line cap=round] (2.1875,17.0625) -- (2.40625,17.28125); 
 \draw [line width=4pt, line cap=round] (2.625,19.25) -- (2.1875,19.6875); 
 \draw [line width=4pt, line cap=round] (2.1875,19.6875) -- (1.3125,19.6875); 
 \draw [line width=4pt, line cap=round] (1.3125,19.6875) -- (1.3125,18.8125); 
 \draw [line width=4pt, line cap=round] (1.3125,18.8125) -- (1.53125,18.59375); 
 \draw [line width=4pt, line cap=round] (1.75,18.375) -- (2.1875,18.8125); 
 \draw [line width=4pt, line cap=round] (1.3125,20.5625) -- (2.1875,20.5625); 
 \draw [line width=4pt, line cap=round] (3.0625,0.4375) -- (3.0625,1.3125); 
 \draw [line width=4pt, line cap=round] (2.1875,0.4375) -- (3.0625,0.4375); 
 \draw [line width=4pt, line cap=round] (3.0625,2.1875) -- (3.0625,3.0625); 
 \draw [line width=4pt, line cap=round] (2.625,3.5) -- (3.0625,3.0625); 
 \draw [line width=4pt, line cap=round] (1.75,2.625) -- (2.1875,3.0625); 
 \draw [line width=4pt, line cap=round] (2.1875,3.0625) -- (2.40625,3.28125); 
 \draw [line width=4pt, line cap=round] (3.0625,3.9375) -- (3.0625,4.8125); 
 \draw [line width=4pt, line cap=round] (3.0625,3.9375) -- (2.84375,3.71875); 
 \draw [line width=4pt, line cap=round] (2.625,5.25) -- (3.0625,4.8125); 
 \draw [line width=4pt, line cap=round] (2.625,3.5) -- (2.1875,3.9375); 
 \draw [line width=4pt, line cap=round] (3.0625,5.6875) -- (3.0625,6.5625); 
 \draw [line width=4pt, line cap=round] (3.0625,5.6875) -- (2.84375,5.46875); 
 \draw [line width=4pt, line cap=round] (2.625,7) -- (3.0625,6.5625); 
 \draw [line width=4pt, line cap=round] (1.75,6.125) -- (2.1875,6.5625); 
 \draw [line width=4pt, line cap=round] (2.1875,6.5625) -- (2.40625,6.78125); 
 \draw [line width=4pt, line cap=round] (2.625,5.25) -- (2.1875,5.6875); 
 \draw [line width=4pt, line cap=round] (2.1875,5.6875) -- (1.96875,5.90625); 
 \draw [line width=4pt, line cap=round] (3.0625,7.4375) -- (3.0625,8.3125); 
 \draw [line width=4pt, line cap=round] (3.0625,7.4375) -- (2.84375,7.21875); 
 \draw [line width=4pt, line cap=round] (2.625,8.75) -- (3.0625,8.3125); 
 \draw [line width=4pt, line cap=round] (1.75,7.875) -- (2.1875,8.3125); 
 \draw [line width=4pt, line cap=round] (2.1875,8.3125) -- (2.40625,8.53125); 
 \draw [line width=4pt, line cap=round] (2.625,7) -- (2.1875,7.4375); 
 \draw [line width=4pt, line cap=round] (2.1875,7.4375) -- (1.96875,7.65625); 
 \draw [line width=4pt, line cap=round] (3.0625,9.1875) -- (3.0625,10.0625); 
 \draw [line width=4pt, line cap=round] (3.0625,9.1875) -- (2.84375,8.96875); 
 \draw [line width=4pt, line cap=round] (2.625,10.5) -- (3.0625,10.0625); 
 \draw [line width=4pt, line cap=round] (1.75,9.625) -- (2.1875,10.0625); 
 \draw [line width=4pt, line cap=round] (2.1875,10.0625) -- (2.40625,10.28125); 
 \draw [line width=4pt, line cap=round] (2.625,8.75) -- (2.1875,9.1875); 
 \draw [line width=4pt, line cap=round] (2.1875,9.1875) -- (1.96875,9.40625); 
 \draw [line width=4pt, line cap=round] (3.0625,10.9375) -- (3.0625,11.8125); 
 \draw [line width=4pt, line cap=round] (3.0625,10.9375) -- (2.84375,10.71875); 
 \draw [line width=4pt, line cap=round] (1.75,11.375) -- (2.1875,11.8125); 
 \draw [line width=4pt, line cap=round] (2.625,10.5) -- (2.1875,10.9375); 
 \draw [line width=4pt, line cap=round] (2.1875,10.9375) -- (1.96875,11.15625); 
 \draw [line width=4pt, line cap=round] (3.0625,12.6875) -- (3.0625,13.5625); 
 \draw [line width=4pt, line cap=round] (2.625,14) -- (3.0625,13.5625); 
 \draw [line width=4pt, line cap=round] (1.75,13.125) -- (2.1875,13.5625); 
 \draw [line width=4pt, line cap=round] (2.1875,13.5625) -- (2.40625,13.78125); 
 \draw [line width=4pt, line cap=round] (3.0625,14.4375) -- (3.0625,15.3125); 
 \draw [line width=4pt, line cap=round] (3.0625,14.4375) -- (2.84375,14.21875); 
 \draw [line width=4pt, line cap=round] (2.625,15.75) -- (3.0625,15.3125); 
 \draw [line width=4pt, line cap=round] (1.75,14.875) -- (2.1875,15.3125); 
 \draw [line width=4pt, line cap=round] (2.1875,15.3125) -- (2.40625,15.53125); 
 \draw [line width=4pt, line cap=round] (2.625,14) -- (2.1875,14.4375); 
 \draw [line width=4pt, line cap=round] (2.1875,14.4375) -- (1.96875,14.65625); 
 \draw [line width=4pt, line cap=round] (3.0625,16.1875) -- (3.0625,17.0625); 
 \draw [line width=4pt, line cap=round] (3.0625,16.1875) -- (2.84375,15.96875); 
 \draw [line width=4pt, line cap=round] (2.625,17.5) -- (3.0625,17.0625); 
 \draw [line width=4pt, line cap=round] (2.625,15.75) -- (2.1875,16.1875); 
 \draw [line width=4pt, line cap=round] (3.0625,17.9375) -- (3.0625,18.8125); 
 \draw [line width=4pt, line cap=round] (3.0625,17.9375) -- (2.84375,17.71875); 
 \draw [line width=4pt, line cap=round] (2.625,19.25) -- (3.0625,18.8125); 
 \draw [line width=4pt, line cap=round] (1.75,18.375) -- (2.1875,18.8125); 
 \draw [line width=4pt, line cap=round] (2.1875,18.8125) -- (2.40625,19.03125); 
 \draw [line width=4pt, line cap=round] (2.625,17.5) -- (2.1875,17.9375); 
 \draw [line width=4pt, line cap=round] (2.1875,17.9375) -- (1.96875,18.15625); 
 \draw [line width=4pt, line cap=round] (3.0625,19.6875) -- (3.0625,20.5625); 
 \draw [line width=4pt, line cap=round] (3.0625,19.6875) -- (2.84375,19.46875); 
 \draw [line width=4pt, line cap=round] (3.0625,20.5625) -- (2.1875,20.5625); 
 \draw [line width=4pt, line cap=round] (2.625,19.25) -- (2.1875,19.6875); 
 \draw [line width=4pt, line cap=round] (3.0625,2.1875) -- (3.0625,1.3125); 
 \draw [line width=4pt, line cap=round] (2.625,3.5) -- (3.0625,3.0625); 
 \draw [line width=4pt, line cap=round] (2.625,5.25) -- (3.0625,4.8125); 
 \draw [line width=4pt, line cap=round] (2.625,7) -- (3.0625,6.5625); 
 \draw [line width=4pt, line cap=round] (2.625,8.75) -- (3.0625,8.3125); 
 \draw [line width=4pt, line cap=round] (2.625,10.5) -- (3.0625,10.0625); 
 \draw [line width=4pt, line cap=round] (3.0625,12.6875) -- (3.0625,11.8125); 
 \draw [line width=4pt, line cap=round] (2.625,14) -- (3.0625,13.5625); 
 \draw [line width=4pt, line cap=round] (2.625,15.75) -- (3.0625,15.3125); 
 \draw [line width=4pt, line cap=round] (2.625,17.5) -- (3.0625,17.0625); 
 \draw [line width=4pt, line cap=round] (2.625,19.25) -- (3.0625,18.8125); 
\end{tikzpicture} 
}
\end{flushright}
\end{marginfigure}
\vspace{0.5cm}

\begin{center}
\input{ch0_generated_files/oneXone}
\end{center}

\vspace{0.5cm}


\noindent
In a 1$\times$2 grid, there is one crossing space, allowing for three possible options, a crossing, a horizontal barrier blocking the crossing, or a vertical barrier blocking the crossing.

\vspace{0.5cm}

\begin{center}
\input{ch0_generated_files/oneXTwoA}\hspace{0.5cm}
\input{ch0_generated_files/oneXTwoB}\hspace{0.5cm}
\input{ch0_generated_files/oneXTwoC}
\end{center}

\vspace{0.5cm}

\noindent
In a 2$\times$2 grid, there are four crossing spaces, with three options (crossing, horizontal barrier, and vertical barrier), providing 3$^4=$81 possible patterns, shown on the next page. These range between the double link that has four crossings, to four distinct loops that have no crossings.

\vspace{0.5cm}
\begin{center}
\input{ch1_generated_files/0000}
\hspace{0.5cm}
\input{ch2_generated_files/2121}
\end{center}

\newpage

\marginnote[+5cm]{There are 81 possible of 2$\times$2 grid patterns, some of which are rotations\\ or reflections of each other.}

\noindent
\scalebox{0.6}{
\begin{minipage}{18cm}
\input{ch2_formatted}
\end{minipage}
}
