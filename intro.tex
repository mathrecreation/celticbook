
In a 1$\times$1 grid, there are no crossing spaces, so only one possible pattern can be drawn, the trivial one-unit loop.

\vspace{0.5cm}

\begin{center}
\begin{tikzpicture}[framed,background rectangle/.style={ultra thick,draw=black}]
 \draw [line width=3pt, line cap=round] (0.75,0.25) -- (0.75,0.75); 
 \draw [line width=3pt, line cap=round] (0.75,0.75) -- (0.25,0.75); 
 \draw [line width=3pt, line cap=round] (0.25,0.75) -- (0.25,0.25); 
 \draw [line width=3pt, line cap=round] (0.25,0.25) -- (0.75,0.25); 
\end{tikzpicture} 

\end{center}

\vspace{0.5cm}


\noindent
In a 1$\times$2 grid, there is one crossing space, allowing for three possible options, a crossing, a horizontal barrier blocking the crossing, or a vertical barrier blocking the crossing.

\vspace{0.5cm}

\begin{center}
\begin{tikzpicture}[framed,background rectangle/.style={ultra thick,draw=black}]
 \draw [line width=3pt, line cap=round] (1,0.5) -- (0.75,0.25); 
 \draw [line width=3pt, line cap=round] (0.75,0.75) -- (0.25,0.75); 
 \draw [line width=3pt, line cap=round] (0.75,0.75) -- (0.875,0.625); 
 \draw [line width=3pt, line cap=round] (0.25,0.75) -- (0.25,0.25); 
 \draw [line width=3pt, line cap=round] (0.25,0.25) -- (0.75,0.25); 
 \draw [line width=3pt, line cap=round] (1,0.5) -- (0.75,0.25); 
 \draw [line width=3pt, line cap=round] (1,0.5) -- (1.25,0.75); 
 \draw [line width=3pt, line cap=round] (1.75,0.25) -- (1.75,0.75); 
 \draw [line width=3pt, line cap=round] (1.75,0.75) -- (1.25,0.75); 
 \draw [line width=3pt, line cap=round] (1,0.5) -- (1.25,0.75); 
 \draw [line width=3pt, line cap=round] (1.25,0.25) -- (1.75,0.25); 
 \draw [line width=3pt, line cap=round] (1.25,0.25) -- (1.125,0.375); 
\end{tikzpicture} 
\hspace{0.5cm}
\begin{tikzpicture}[framed,background rectangle/.style={ultra thick,draw=black}]
 \draw [line width=3pt, line cap=round] (0.75,0.75) -- (0.25,0.75); 
 \draw [line width=3pt, line cap=round] (0.25,0.75) -- (0.25,0.25); 
 \draw [line width=3pt, line cap=round] (0.25,0.25) -- (0.75,0.25); 
 \draw [line width=3pt, line cap=round] (1.25,0.25) -- (0.75,0.25); 
 \draw [line width=3pt, line cap=round] (0.75,0.75) -- (1.25,0.75); 
 \draw [line width=3pt, line cap=round] (1.75,0.25) -- (1.75,0.75); 
 \draw [line width=3pt, line cap=round] (1.75,0.75) -- (1.25,0.75); 
 \draw [line width=3pt, line cap=round] (1.25,0.25) -- (1.75,0.25); 
\end{tikzpicture} 
\hspace{0.5cm}
\begin{tikzpicture}[framed,background rectangle/.style={ultra thick,draw=black}]
 \draw [line width=3pt, line cap=round] (0.75,0.25) -- (0.75,0.75); 
 \draw [line width=3pt, line cap=round] (0.75,0.75) -- (0.25,0.75); 
 \draw [line width=3pt, line cap=round] (0.25,0.75) -- (0.25,0.25); 
 \draw [line width=3pt, line cap=round] (0.25,0.25) -- (0.75,0.25); 
 \draw [line width=3pt, line cap=round] (1.75,0.25) -- (1.75,0.75); 
 \draw [line width=3pt, line cap=round] (1.75,0.75) -- (1.25,0.75); 
 \draw [line width=3pt, line cap=round] (1.25,0.75) -- (1.25,0.25); 
 \draw [line width=3pt, line cap=round] (1.25,0.25) -- (1.75,0.25); 
\end{tikzpicture} 

\end{center}

\vspace{0.5cm}

\noindent
In a 2$\times$2 grid, there are four crossing spaces, with three options (crossing, horizontal barrier, and vertical barrier), providing 3$^4=$81 possible patterns, shown on the next page.

\newpage

\marginnote[+5cm]{There are 81 possible of 2$\times$2 grid patterns, some of which are rotations\\ or reflections of each other.}

\noindent
\scalebox{0.6}{
\begin{minipage}{18cm}
\noindent
\begin{tikzpicture}[framed,background rectangle/.style={ultra thick,draw=black}]
 \draw [line width=3pt, line cap=round] (0.5,1) -- (0.25,1.25); 
 \draw [line width=3pt, line cap=round] (1,0.5) -- (0.75,0.25); 
 \draw [line width=3pt, line cap=round] (0.5,1) -- (0.75,0.75); 
 \draw [line width=3pt, line cap=round] (0.75,0.75) -- (0.875,0.625); 
 \draw [line width=3pt, line cap=round] (0.25,0.75) -- (0.25,0.25); 
 \draw [line width=3pt, line cap=round] (0.25,0.75) -- (0.375,0.875); 
 \draw [line width=3pt, line cap=round] (0.25,0.25) -- (0.75,0.25); 
 \draw [line width=3pt, line cap=round] (1,1.5) -- (0.75,1.25); 
 \draw [line width=3pt, line cap=round] (0.75,1.25) -- (0.625,1.125); 
 \draw [line width=3pt, line cap=round] (0.75,1.75) -- (0.25,1.75); 
 \draw [line width=3pt, line cap=round] (0.75,1.75) -- (0.875,1.625); 
 \draw [line width=3pt, line cap=round] (0.25,1.75) -- (0.25,1.25); 
 \draw [line width=3pt, line cap=round] (0.5,1) -- (0.25,1.25); 
 \draw [line width=3pt, line cap=round] (1,0.5) -- (0.75,0.25); 
 \draw [line width=3pt, line cap=round] (1.5,1) -- (1.25,1.25); 
 \draw [line width=3pt, line cap=round] (1,1.5) -- (0.75,1.25); 
 \draw [line width=3pt, line cap=round] (0.5,1) -- (0.75,0.75); 
 \draw [line width=3pt, line cap=round] (1,0.5) -- (1.25,0.75); 
 \draw [line width=3pt, line cap=round] (1,1.5) -- (1.25,1.75); 
 \draw [line width=3pt, line cap=round] (1.75,0.25) -- (1.75,0.75); 
 \draw [line width=3pt, line cap=round] (1.5,1) -- (1.75,0.75); 
 \draw [line width=3pt, line cap=round] (1,0.5) -- (1.25,0.75); 
 \draw [line width=3pt, line cap=round] (1.25,0.75) -- (1.375,0.875); 
 \draw [line width=3pt, line cap=round] (1.25,0.25) -- (1.75,0.25); 
 \draw [line width=3pt, line cap=round] (1.25,0.25) -- (1.125,0.375); 
 \draw [line width=3pt, line cap=round] (1.75,1.25) -- (1.75,1.75); 
 \draw [line width=3pt, line cap=round] (1.75,1.25) -- (1.625,1.125); 
 \draw [line width=3pt, line cap=round] (1.75,1.75) -- (1.25,1.75); 
 \draw [line width=3pt, line cap=round] (1,1.5) -- (1.25,1.75); 
 \draw [line width=3pt, line cap=round] (1.5,1) -- (1.25,1.25); 
 \draw [line width=3pt, line cap=round] (1.25,1.25) -- (1.125,1.375); 
 \draw [line width=3pt, line cap=round] (1.5,1) -- (1.75,0.75); 
\end{tikzpicture} 

\begin{tikzpicture}[framed,background rectangle/.style={ultra thick,draw=black}]
 \draw [line width=3pt, line cap=round] (1,0.5) -- (0.75,0.25); 
 \draw [line width=3pt, line cap=round] (0.75,0.75) -- (0.25,0.75); 
 \draw [line width=3pt, line cap=round] (0.75,0.75) -- (0.875,0.625); 
 \draw [line width=3pt, line cap=round] (0.25,0.75) -- (0.25,0.25); 
 \draw [line width=3pt, line cap=round] (0.25,0.25) -- (0.75,0.25); 
 \draw [line width=3pt, line cap=round] (1,1.5) -- (0.75,1.25); 
 \draw [line width=3pt, line cap=round] (0.75,1.75) -- (0.25,1.75); 
 \draw [line width=3pt, line cap=round] (0.75,1.75) -- (0.875,1.625); 
 \draw [line width=3pt, line cap=round] (0.25,1.75) -- (0.25,1.25); 
 \draw [line width=3pt, line cap=round] (0.25,1.25) -- (0.75,1.25); 
 \draw [line width=3pt, line cap=round] (1,0.5) -- (0.75,0.25); 
 \draw [line width=3pt, line cap=round] (1.5,1) -- (1.25,1.25); 
 \draw [line width=3pt, line cap=round] (1,1.5) -- (0.75,1.25); 
 \draw [line width=3pt, line cap=round] (1,0.5) -- (1.25,0.75); 
 \draw [line width=3pt, line cap=round] (1,1.5) -- (1.25,1.75); 
 \draw [line width=3pt, line cap=round] (1.75,0.25) -- (1.75,0.75); 
 \draw [line width=3pt, line cap=round] (1.5,1) -- (1.75,0.75); 
 \draw [line width=3pt, line cap=round] (1,0.5) -- (1.25,0.75); 
 \draw [line width=3pt, line cap=round] (1.25,0.75) -- (1.375,0.875); 
 \draw [line width=3pt, line cap=round] (1.25,0.25) -- (1.75,0.25); 
 \draw [line width=3pt, line cap=round] (1.25,0.25) -- (1.125,0.375); 
 \draw [line width=3pt, line cap=round] (1.75,1.25) -- (1.75,1.75); 
 \draw [line width=3pt, line cap=round] (1.75,1.25) -- (1.625,1.125); 
 \draw [line width=3pt, line cap=round] (1.75,1.75) -- (1.25,1.75); 
 \draw [line width=3pt, line cap=round] (1,1.5) -- (1.25,1.75); 
 \draw [line width=3pt, line cap=round] (1.5,1) -- (1.25,1.25); 
 \draw [line width=3pt, line cap=round] (1.25,1.25) -- (1.125,1.375); 
 \draw [line width=3pt, line cap=round] (1.5,1) -- (1.75,0.75); 
\end{tikzpicture} 

\begin{tikzpicture}[framed,background rectangle/.style={ultra thick,draw=black}]
 \draw [line width=3pt, line cap=round] (0.25,0.75) -- (0.25,1.25); 
 \draw [line width=3pt, line cap=round] (1,0.5) -- (0.75,0.25); 
 \draw [line width=3pt, line cap=round] (0.25,0.75) -- (0.25,0.25); 
 \draw [line width=3pt, line cap=round] (0.25,0.25) -- (0.75,0.25); 
 \draw [line width=3pt, line cap=round] (1,1.5) -- (0.75,1.25); 
 \draw [line width=3pt, line cap=round] (0.75,1.75) -- (0.25,1.75); 
 \draw [line width=3pt, line cap=round] (0.75,1.75) -- (0.875,1.625); 
 \draw [line width=3pt, line cap=round] (0.25,1.75) -- (0.25,1.25); 
 \draw [line width=3pt, line cap=round] (1,0.5) -- (0.75,0.25); 
 \draw [line width=3pt, line cap=round] (1.5,1) -- (1.25,1.25); 
 \draw [line width=3pt, line cap=round] (1,1.5) -- (0.75,1.25); 
 \draw [line width=3pt, line cap=round] (0.75,1.25) -- (0.75,0.75); 
 \draw [line width=3pt, line cap=round] (0.75,0.75) -- (0.875,0.625); 
 \draw [line width=3pt, line cap=round] (1,0.5) -- (1.25,0.75); 
 \draw [line width=3pt, line cap=round] (1,1.5) -- (1.25,1.75); 
 \draw [line width=3pt, line cap=round] (1.75,0.25) -- (1.75,0.75); 
 \draw [line width=3pt, line cap=round] (1.5,1) -- (1.75,0.75); 
 \draw [line width=3pt, line cap=round] (1,0.5) -- (1.25,0.75); 
 \draw [line width=3pt, line cap=round] (1.25,0.75) -- (1.375,0.875); 
 \draw [line width=3pt, line cap=round] (1.25,0.25) -- (1.75,0.25); 
 \draw [line width=3pt, line cap=round] (1.25,0.25) -- (1.125,0.375); 
 \draw [line width=3pt, line cap=round] (1.75,1.25) -- (1.75,1.75); 
 \draw [line width=3pt, line cap=round] (1.75,1.25) -- (1.625,1.125); 
 \draw [line width=3pt, line cap=round] (1.75,1.75) -- (1.25,1.75); 
 \draw [line width=3pt, line cap=round] (1,1.5) -- (1.25,1.75); 
 \draw [line width=3pt, line cap=round] (1.5,1) -- (1.25,1.25); 
 \draw [line width=3pt, line cap=round] (1.25,1.25) -- (1.125,1.375); 
 \draw [line width=3pt, line cap=round] (1.5,1) -- (1.75,0.75); 
\end{tikzpicture} 

\begin{tikzpicture}[framed,background rectangle/.style={ultra thick,draw=black}]
 \draw [line width=3pt, line cap=round] (0.5,1) -- (0.25,1.25); 
 \draw [line width=3pt, line cap=round] (1,0.5) -- (0.75,0.25); 
 \draw [line width=3pt, line cap=round] (0.5,1) -- (0.75,0.75); 
 \draw [line width=3pt, line cap=round] (0.75,0.75) -- (0.875,0.625); 
 \draw [line width=3pt, line cap=round] (0.25,0.75) -- (0.25,0.25); 
 \draw [line width=3pt, line cap=round] (0.25,0.75) -- (0.375,0.875); 
 \draw [line width=3pt, line cap=round] (0.25,0.25) -- (0.75,0.25); 
 \draw [line width=3pt, line cap=round] (0.75,1.75) -- (0.25,1.75); 
 \draw [line width=3pt, line cap=round] (0.25,1.75) -- (0.25,1.25); 
 \draw [line width=3pt, line cap=round] (0.5,1) -- (0.25,1.25); 
 \draw [line width=3pt, line cap=round] (1,0.5) -- (0.75,0.25); 
 \draw [line width=3pt, line cap=round] (1.5,1) -- (1.25,1.25); 
 \draw [line width=3pt, line cap=round] (1.25,1.25) -- (0.75,1.25); 
 \draw [line width=3pt, line cap=round] (0.75,1.25) -- (0.625,1.125); 
 \draw [line width=3pt, line cap=round] (0.5,1) -- (0.75,0.75); 
 \draw [line width=3pt, line cap=round] (1,0.5) -- (1.25,0.75); 
 \draw [line width=3pt, line cap=round] (0.75,1.75) -- (1.25,1.75); 
 \draw [line width=3pt, line cap=round] (1.75,0.25) -- (1.75,0.75); 
 \draw [line width=3pt, line cap=round] (1.5,1) -- (1.75,0.75); 
 \draw [line width=3pt, line cap=round] (1,0.5) -- (1.25,0.75); 
 \draw [line width=3pt, line cap=round] (1.25,0.75) -- (1.375,0.875); 
 \draw [line width=3pt, line cap=round] (1.25,0.25) -- (1.75,0.25); 
 \draw [line width=3pt, line cap=round] (1.25,0.25) -- (1.125,0.375); 
 \draw [line width=3pt, line cap=round] (1.75,1.25) -- (1.75,1.75); 
 \draw [line width=3pt, line cap=round] (1.75,1.25) -- (1.625,1.125); 
 \draw [line width=3pt, line cap=round] (1.75,1.75) -- (1.25,1.75); 
 \draw [line width=3pt, line cap=round] (1.5,1) -- (1.25,1.25); 
 \draw [line width=3pt, line cap=round] (1.5,1) -- (1.75,0.75); 
\end{tikzpicture} 

\begin{tikzpicture}[framed,background rectangle/.style={ultra thick,draw=black}]
 \draw [line width=3pt, line cap=round] (1,0.5) -- (0.75,0.25); 
 \draw [line width=3pt, line cap=round] (0.75,0.75) -- (0.25,0.75); 
 \draw [line width=3pt, line cap=round] (0.75,0.75) -- (0.875,0.625); 
 \draw [line width=3pt, line cap=round] (0.25,0.75) -- (0.25,0.25); 
 \draw [line width=3pt, line cap=round] (0.25,0.25) -- (0.75,0.25); 
 \draw [line width=3pt, line cap=round] (0.75,1.75) -- (0.25,1.75); 
 \draw [line width=3pt, line cap=round] (0.25,1.75) -- (0.25,1.25); 
 \draw [line width=3pt, line cap=round] (0.25,1.25) -- (0.75,1.25); 
 \draw [line width=3pt, line cap=round] (1,0.5) -- (0.75,0.25); 
 \draw [line width=3pt, line cap=round] (1.5,1) -- (1.25,1.25); 
 \draw [line width=3pt, line cap=round] (1.25,1.25) -- (0.75,1.25); 
 \draw [line width=3pt, line cap=round] (1,0.5) -- (1.25,0.75); 
 \draw [line width=3pt, line cap=round] (0.75,1.75) -- (1.25,1.75); 
 \draw [line width=3pt, line cap=round] (1.75,0.25) -- (1.75,0.75); 
 \draw [line width=3pt, line cap=round] (1.5,1) -- (1.75,0.75); 
 \draw [line width=3pt, line cap=round] (1,0.5) -- (1.25,0.75); 
 \draw [line width=3pt, line cap=round] (1.25,0.75) -- (1.375,0.875); 
 \draw [line width=3pt, line cap=round] (1.25,0.25) -- (1.75,0.25); 
 \draw [line width=3pt, line cap=round] (1.25,0.25) -- (1.125,0.375); 
 \draw [line width=3pt, line cap=round] (1.75,1.25) -- (1.75,1.75); 
 \draw [line width=3pt, line cap=round] (1.75,1.25) -- (1.625,1.125); 
 \draw [line width=3pt, line cap=round] (1.75,1.75) -- (1.25,1.75); 
 \draw [line width=3pt, line cap=round] (1.5,1) -- (1.25,1.25); 
 \draw [line width=3pt, line cap=round] (1.5,1) -- (1.75,0.75); 
\end{tikzpicture} 


\noindent
\begin{tikzpicture}[framed,background rectangle/.style={ultra thick,draw=black}]
 \draw [line width=3pt, line cap=round] (0.25,0.75) -- (0.25,1.25); 
 \draw [line width=3pt, line cap=round] (1,0.5) -- (0.75,0.25); 
 \draw [line width=3pt, line cap=round] (0.25,0.75) -- (0.25,0.25); 
 \draw [line width=3pt, line cap=round] (0.25,0.25) -- (0.75,0.25); 
 \draw [line width=3pt, line cap=round] (0.75,1.75) -- (0.25,1.75); 
 \draw [line width=3pt, line cap=round] (0.25,1.75) -- (0.25,1.25); 
 \draw [line width=3pt, line cap=round] (1,0.5) -- (0.75,0.25); 
 \draw [line width=3pt, line cap=round] (1.5,1) -- (1.25,1.25); 
 \draw [line width=3pt, line cap=round] (1.25,1.25) -- (0.75,1.25); 
 \draw [line width=3pt, line cap=round] (0.75,1.25) -- (0.75,0.75); 
 \draw [line width=3pt, line cap=round] (0.75,0.75) -- (0.875,0.625); 
 \draw [line width=3pt, line cap=round] (1,0.5) -- (1.25,0.75); 
 \draw [line width=3pt, line cap=round] (0.75,1.75) -- (1.25,1.75); 
 \draw [line width=3pt, line cap=round] (1.75,0.25) -- (1.75,0.75); 
 \draw [line width=3pt, line cap=round] (1.5,1) -- (1.75,0.75); 
 \draw [line width=3pt, line cap=round] (1,0.5) -- (1.25,0.75); 
 \draw [line width=3pt, line cap=round] (1.25,0.75) -- (1.375,0.875); 
 \draw [line width=3pt, line cap=round] (1.25,0.25) -- (1.75,0.25); 
 \draw [line width=3pt, line cap=round] (1.25,0.25) -- (1.125,0.375); 
 \draw [line width=3pt, line cap=round] (1.75,1.25) -- (1.75,1.75); 
 \draw [line width=3pt, line cap=round] (1.75,1.25) -- (1.625,1.125); 
 \draw [line width=3pt, line cap=round] (1.75,1.75) -- (1.25,1.75); 
 \draw [line width=3pt, line cap=round] (1.5,1) -- (1.25,1.25); 
 \draw [line width=3pt, line cap=round] (1.5,1) -- (1.75,0.75); 
\end{tikzpicture} 

\begin{tikzpicture}[framed,background rectangle/.style={ultra thick,draw=black}]
 \draw [line width=3pt, line cap=round] (0.5,1) -- (0.25,1.25); 
 \draw [line width=3pt, line cap=round] (1,0.5) -- (0.75,0.25); 
 \draw [line width=3pt, line cap=round] (0.5,1) -- (0.75,0.75); 
 \draw [line width=3pt, line cap=round] (0.75,0.75) -- (0.875,0.625); 
 \draw [line width=3pt, line cap=round] (0.25,0.75) -- (0.25,0.25); 
 \draw [line width=3pt, line cap=round] (0.25,0.75) -- (0.375,0.875); 
 \draw [line width=3pt, line cap=round] (0.25,0.25) -- (0.75,0.25); 
 \draw [line width=3pt, line cap=round] (0.75,1.25) -- (0.75,1.75); 
 \draw [line width=3pt, line cap=round] (0.75,1.25) -- (0.625,1.125); 
 \draw [line width=3pt, line cap=round] (0.75,1.75) -- (0.25,1.75); 
 \draw [line width=3pt, line cap=round] (0.25,1.75) -- (0.25,1.25); 
 \draw [line width=3pt, line cap=round] (0.5,1) -- (0.25,1.25); 
 \draw [line width=3pt, line cap=round] (1,0.5) -- (0.75,0.25); 
 \draw [line width=3pt, line cap=round] (1.5,1) -- (1.25,1.25); 
 \draw [line width=3pt, line cap=round] (0.5,1) -- (0.75,0.75); 
 \draw [line width=3pt, line cap=round] (1,0.5) -- (1.25,0.75); 
 \draw [line width=3pt, line cap=round] (1.75,0.25) -- (1.75,0.75); 
 \draw [line width=3pt, line cap=round] (1.5,1) -- (1.75,0.75); 
 \draw [line width=3pt, line cap=round] (1,0.5) -- (1.25,0.75); 
 \draw [line width=3pt, line cap=round] (1.25,0.75) -- (1.375,0.875); 
 \draw [line width=3pt, line cap=round] (1.25,0.25) -- (1.75,0.25); 
 \draw [line width=3pt, line cap=round] (1.25,0.25) -- (1.125,0.375); 
 \draw [line width=3pt, line cap=round] (1.75,1.25) -- (1.75,1.75); 
 \draw [line width=3pt, line cap=round] (1.75,1.25) -- (1.625,1.125); 
 \draw [line width=3pt, line cap=round] (1.75,1.75) -- (1.25,1.75); 
 \draw [line width=3pt, line cap=round] (1.25,1.75) -- (1.25,1.25); 
 \draw [line width=3pt, line cap=round] (1.5,1) -- (1.25,1.25); 
 \draw [line width=3pt, line cap=round] (1.5,1) -- (1.75,0.75); 
\end{tikzpicture} 

\begin{tikzpicture}[framed,background rectangle/.style={ultra thick,draw=black}]
 \draw [line width=3pt, line cap=round] (1,0.5) -- (0.75,0.25); 
 \draw [line width=3pt, line cap=round] (0.75,0.75) -- (0.25,0.75); 
 \draw [line width=3pt, line cap=round] (0.75,0.75) -- (0.875,0.625); 
 \draw [line width=3pt, line cap=round] (0.25,0.75) -- (0.25,0.25); 
 \draw [line width=3pt, line cap=round] (0.25,0.25) -- (0.75,0.25); 
 \draw [line width=3pt, line cap=round] (0.75,1.25) -- (0.75,1.75); 
 \draw [line width=3pt, line cap=round] (0.75,1.75) -- (0.25,1.75); 
 \draw [line width=3pt, line cap=round] (0.25,1.75) -- (0.25,1.25); 
 \draw [line width=3pt, line cap=round] (0.25,1.25) -- (0.75,1.25); 
 \draw [line width=3pt, line cap=round] (1,0.5) -- (0.75,0.25); 
 \draw [line width=3pt, line cap=round] (1.5,1) -- (1.25,1.25); 
 \draw [line width=3pt, line cap=round] (1,0.5) -- (1.25,0.75); 
 \draw [line width=3pt, line cap=round] (1.75,0.25) -- (1.75,0.75); 
 \draw [line width=3pt, line cap=round] (1.5,1) -- (1.75,0.75); 
 \draw [line width=3pt, line cap=round] (1,0.5) -- (1.25,0.75); 
 \draw [line width=3pt, line cap=round] (1.25,0.75) -- (1.375,0.875); 
 \draw [line width=3pt, line cap=round] (1.25,0.25) -- (1.75,0.25); 
 \draw [line width=3pt, line cap=round] (1.25,0.25) -- (1.125,0.375); 
 \draw [line width=3pt, line cap=round] (1.75,1.25) -- (1.75,1.75); 
 \draw [line width=3pt, line cap=round] (1.75,1.25) -- (1.625,1.125); 
 \draw [line width=3pt, line cap=round] (1.75,1.75) -- (1.25,1.75); 
 \draw [line width=3pt, line cap=round] (1.25,1.75) -- (1.25,1.25); 
 \draw [line width=3pt, line cap=round] (1.5,1) -- (1.25,1.25); 
 \draw [line width=3pt, line cap=round] (1.5,1) -- (1.75,0.75); 
\end{tikzpicture} 

\begin{tikzpicture}[framed,background rectangle/.style={ultra thick,draw=black}]
 \draw [line width=3pt, line cap=round] (0.25,0.75) -- (0.25,1.25); 
 \draw [line width=3pt, line cap=round] (1,0.5) -- (0.75,0.25); 
 \draw [line width=3pt, line cap=round] (0.25,0.75) -- (0.25,0.25); 
 \draw [line width=3pt, line cap=round] (0.25,0.25) -- (0.75,0.25); 
 \draw [line width=3pt, line cap=round] (0.75,1.25) -- (0.75,1.75); 
 \draw [line width=3pt, line cap=round] (0.75,1.75) -- (0.25,1.75); 
 \draw [line width=3pt, line cap=round] (0.25,1.75) -- (0.25,1.25); 
 \draw [line width=3pt, line cap=round] (1,0.5) -- (0.75,0.25); 
 \draw [line width=3pt, line cap=round] (1.5,1) -- (1.25,1.25); 
 \draw [line width=3pt, line cap=round] (0.75,1.25) -- (0.75,0.75); 
 \draw [line width=3pt, line cap=round] (0.75,0.75) -- (0.875,0.625); 
 \draw [line width=3pt, line cap=round] (1,0.5) -- (1.25,0.75); 
 \draw [line width=3pt, line cap=round] (1.75,0.25) -- (1.75,0.75); 
 \draw [line width=3pt, line cap=round] (1.5,1) -- (1.75,0.75); 
 \draw [line width=3pt, line cap=round] (1,0.5) -- (1.25,0.75); 
 \draw [line width=3pt, line cap=round] (1.25,0.75) -- (1.375,0.875); 
 \draw [line width=3pt, line cap=round] (1.25,0.25) -- (1.75,0.25); 
 \draw [line width=3pt, line cap=round] (1.25,0.25) -- (1.125,0.375); 
 \draw [line width=3pt, line cap=round] (1.75,1.25) -- (1.75,1.75); 
 \draw [line width=3pt, line cap=round] (1.75,1.25) -- (1.625,1.125); 
 \draw [line width=3pt, line cap=round] (1.75,1.75) -- (1.25,1.75); 
 \draw [line width=3pt, line cap=round] (1.25,1.75) -- (1.25,1.25); 
 \draw [line width=3pt, line cap=round] (1.5,1) -- (1.25,1.25); 
 \draw [line width=3pt, line cap=round] (1.5,1) -- (1.75,0.75); 
\end{tikzpicture} 

\begin{tikzpicture}[framed,background rectangle/.style={ultra thick,draw=black}]
 \draw [line width=3pt, line cap=round] (0.5,1) -- (0.25,1.25); 
 \draw [line width=3pt, line cap=round] (1,0.5) -- (0.75,0.25); 
 \draw [line width=3pt, line cap=round] (0.5,1) -- (0.75,0.75); 
 \draw [line width=3pt, line cap=round] (0.75,0.75) -- (0.875,0.625); 
 \draw [line width=3pt, line cap=round] (0.25,0.75) -- (0.25,0.25); 
 \draw [line width=3pt, line cap=round] (0.25,0.75) -- (0.375,0.875); 
 \draw [line width=3pt, line cap=round] (0.25,0.25) -- (0.75,0.25); 
 \draw [line width=3pt, line cap=round] (1,1.5) -- (0.75,1.25); 
 \draw [line width=3pt, line cap=round] (0.75,1.25) -- (0.625,1.125); 
 \draw [line width=3pt, line cap=round] (0.75,1.75) -- (0.25,1.75); 
 \draw [line width=3pt, line cap=round] (0.75,1.75) -- (0.875,1.625); 
 \draw [line width=3pt, line cap=round] (0.25,1.75) -- (0.25,1.25); 
 \draw [line width=3pt, line cap=round] (0.5,1) -- (0.25,1.25); 
 \draw [line width=3pt, line cap=round] (1,0.5) -- (0.75,0.25); 
 \draw [line width=3pt, line cap=round] (1,1.5) -- (0.75,1.25); 
 \draw [line width=3pt, line cap=round] (0.5,1) -- (0.75,0.75); 
 \draw [line width=3pt, line cap=round] (1,0.5) -- (1.25,0.75); 
 \draw [line width=3pt, line cap=round] (1,1.5) -- (1.25,1.75); 
 \draw [line width=3pt, line cap=round] (1.75,0.25) -- (1.75,0.75); 
 \draw [line width=3pt, line cap=round] (1.75,0.75) -- (1.25,0.75); 
 \draw [line width=3pt, line cap=round] (1,0.5) -- (1.25,0.75); 
 \draw [line width=3pt, line cap=round] (1.25,0.25) -- (1.75,0.25); 
 \draw [line width=3pt, line cap=round] (1.25,0.25) -- (1.125,0.375); 
 \draw [line width=3pt, line cap=round] (1.75,1.25) -- (1.75,1.75); 
 \draw [line width=3pt, line cap=round] (1.75,1.75) -- (1.25,1.75); 
 \draw [line width=3pt, line cap=round] (1,1.5) -- (1.25,1.75); 
 \draw [line width=3pt, line cap=round] (1.25,1.25) -- (1.75,1.25); 
 \draw [line width=3pt, line cap=round] (1.25,1.25) -- (1.125,1.375); 
\end{tikzpicture} 


\noindent
\begin{tikzpicture}[framed,background rectangle/.style={ultra thick,draw=black}]
 \draw [line width=3pt, line cap=round] (1,0.5) -- (0.75,0.25); 
 \draw [line width=3pt, line cap=round] (0.75,0.75) -- (0.25,0.75); 
 \draw [line width=3pt, line cap=round] (0.75,0.75) -- (0.875,0.625); 
 \draw [line width=3pt, line cap=round] (0.25,0.75) -- (0.25,0.25); 
 \draw [line width=3pt, line cap=round] (0.25,0.25) -- (0.75,0.25); 
 \draw [line width=3pt, line cap=round] (1,1.5) -- (0.75,1.25); 
 \draw [line width=3pt, line cap=round] (0.75,1.75) -- (0.25,1.75); 
 \draw [line width=3pt, line cap=round] (0.75,1.75) -- (0.875,1.625); 
 \draw [line width=3pt, line cap=round] (0.25,1.75) -- (0.25,1.25); 
 \draw [line width=3pt, line cap=round] (0.25,1.25) -- (0.75,1.25); 
 \draw [line width=3pt, line cap=round] (1,0.5) -- (0.75,0.25); 
 \draw [line width=3pt, line cap=round] (1,1.5) -- (0.75,1.25); 
 \draw [line width=3pt, line cap=round] (1,0.5) -- (1.25,0.75); 
 \draw [line width=3pt, line cap=round] (1,1.5) -- (1.25,1.75); 
 \draw [line width=3pt, line cap=round] (1.75,0.25) -- (1.75,0.75); 
 \draw [line width=3pt, line cap=round] (1.75,0.75) -- (1.25,0.75); 
 \draw [line width=3pt, line cap=round] (1,0.5) -- (1.25,0.75); 
 \draw [line width=3pt, line cap=round] (1.25,0.25) -- (1.75,0.25); 
 \draw [line width=3pt, line cap=round] (1.25,0.25) -- (1.125,0.375); 
 \draw [line width=3pt, line cap=round] (1.75,1.25) -- (1.75,1.75); 
 \draw [line width=3pt, line cap=round] (1.75,1.75) -- (1.25,1.75); 
 \draw [line width=3pt, line cap=round] (1,1.5) -- (1.25,1.75); 
 \draw [line width=3pt, line cap=round] (1.25,1.25) -- (1.75,1.25); 
 \draw [line width=3pt, line cap=round] (1.25,1.25) -- (1.125,1.375); 
\end{tikzpicture} 

\begin{tikzpicture}[framed,background rectangle/.style={ultra thick,draw=black}]
 \draw [line width=3pt, line cap=round] (0.25,0.75) -- (0.25,1.25); 
 \draw [line width=3pt, line cap=round] (1,0.5) -- (0.75,0.25); 
 \draw [line width=3pt, line cap=round] (0.25,0.75) -- (0.25,0.25); 
 \draw [line width=3pt, line cap=round] (0.25,0.25) -- (0.75,0.25); 
 \draw [line width=3pt, line cap=round] (1,1.5) -- (0.75,1.25); 
 \draw [line width=3pt, line cap=round] (0.75,1.75) -- (0.25,1.75); 
 \draw [line width=3pt, line cap=round] (0.75,1.75) -- (0.875,1.625); 
 \draw [line width=3pt, line cap=round] (0.25,1.75) -- (0.25,1.25); 
 \draw [line width=3pt, line cap=round] (1,0.5) -- (0.75,0.25); 
 \draw [line width=3pt, line cap=round] (1,1.5) -- (0.75,1.25); 
 \draw [line width=3pt, line cap=round] (0.75,1.25) -- (0.75,0.75); 
 \draw [line width=3pt, line cap=round] (0.75,0.75) -- (0.875,0.625); 
 \draw [line width=3pt, line cap=round] (1,0.5) -- (1.25,0.75); 
 \draw [line width=3pt, line cap=round] (1,1.5) -- (1.25,1.75); 
 \draw [line width=3pt, line cap=round] (1.75,0.25) -- (1.75,0.75); 
 \draw [line width=3pt, line cap=round] (1.75,0.75) -- (1.25,0.75); 
 \draw [line width=3pt, line cap=round] (1,0.5) -- (1.25,0.75); 
 \draw [line width=3pt, line cap=round] (1.25,0.25) -- (1.75,0.25); 
 \draw [line width=3pt, line cap=round] (1.25,0.25) -- (1.125,0.375); 
 \draw [line width=3pt, line cap=round] (1.75,1.25) -- (1.75,1.75); 
 \draw [line width=3pt, line cap=round] (1.75,1.75) -- (1.25,1.75); 
 \draw [line width=3pt, line cap=round] (1,1.5) -- (1.25,1.75); 
 \draw [line width=3pt, line cap=round] (1.25,1.25) -- (1.75,1.25); 
 \draw [line width=3pt, line cap=round] (1.25,1.25) -- (1.125,1.375); 
\end{tikzpicture} 

\begin{tikzpicture}[framed,background rectangle/.style={ultra thick,draw=black}]
 \draw [line width=3pt, line cap=round] (0.5,1) -- (0.25,1.25); 
 \draw [line width=3pt, line cap=round] (1,0.5) -- (0.75,0.25); 
 \draw [line width=3pt, line cap=round] (0.5,1) -- (0.75,0.75); 
 \draw [line width=3pt, line cap=round] (0.75,0.75) -- (0.875,0.625); 
 \draw [line width=3pt, line cap=round] (0.25,0.75) -- (0.25,0.25); 
 \draw [line width=3pt, line cap=round] (0.25,0.75) -- (0.375,0.875); 
 \draw [line width=3pt, line cap=round] (0.25,0.25) -- (0.75,0.25); 
 \draw [line width=3pt, line cap=round] (0.75,1.75) -- (0.25,1.75); 
 \draw [line width=3pt, line cap=round] (0.25,1.75) -- (0.25,1.25); 
 \draw [line width=3pt, line cap=round] (0.5,1) -- (0.25,1.25); 
 \draw [line width=3pt, line cap=round] (1,0.5) -- (0.75,0.25); 
 \draw [line width=3pt, line cap=round] (1.25,1.25) -- (0.75,1.25); 
 \draw [line width=3pt, line cap=round] (0.75,1.25) -- (0.625,1.125); 
 \draw [line width=3pt, line cap=round] (0.5,1) -- (0.75,0.75); 
 \draw [line width=3pt, line cap=round] (1,0.5) -- (1.25,0.75); 
 \draw [line width=3pt, line cap=round] (0.75,1.75) -- (1.25,1.75); 
 \draw [line width=3pt, line cap=round] (1.75,0.25) -- (1.75,0.75); 
 \draw [line width=3pt, line cap=round] (1.75,0.75) -- (1.25,0.75); 
 \draw [line width=3pt, line cap=round] (1,0.5) -- (1.25,0.75); 
 \draw [line width=3pt, line cap=round] (1.25,0.25) -- (1.75,0.25); 
 \draw [line width=3pt, line cap=round] (1.25,0.25) -- (1.125,0.375); 
 \draw [line width=3pt, line cap=round] (1.75,1.25) -- (1.75,1.75); 
 \draw [line width=3pt, line cap=round] (1.75,1.75) -- (1.25,1.75); 
 \draw [line width=3pt, line cap=round] (1.25,1.25) -- (1.75,1.25); 
\end{tikzpicture} 

\begin{tikzpicture}[framed,background rectangle/.style={ultra thick,draw=black}]
 \draw [line width=3pt, line cap=round] (1,0.5) -- (0.75,0.25); 
 \draw [line width=3pt, line cap=round] (0.75,0.75) -- (0.25,0.75); 
 \draw [line width=3pt, line cap=round] (0.75,0.75) -- (0.875,0.625); 
 \draw [line width=3pt, line cap=round] (0.25,0.75) -- (0.25,0.25); 
 \draw [line width=3pt, line cap=round] (0.25,0.25) -- (0.75,0.25); 
 \draw [line width=3pt, line cap=round] (0.75,1.75) -- (0.25,1.75); 
 \draw [line width=3pt, line cap=round] (0.25,1.75) -- (0.25,1.25); 
 \draw [line width=3pt, line cap=round] (0.25,1.25) -- (0.75,1.25); 
 \draw [line width=3pt, line cap=round] (1,0.5) -- (0.75,0.25); 
 \draw [line width=3pt, line cap=round] (1.25,1.25) -- (0.75,1.25); 
 \draw [line width=3pt, line cap=round] (1,0.5) -- (1.25,0.75); 
 \draw [line width=3pt, line cap=round] (0.75,1.75) -- (1.25,1.75); 
 \draw [line width=3pt, line cap=round] (1.75,0.25) -- (1.75,0.75); 
 \draw [line width=3pt, line cap=round] (1.75,0.75) -- (1.25,0.75); 
 \draw [line width=3pt, line cap=round] (1,0.5) -- (1.25,0.75); 
 \draw [line width=3pt, line cap=round] (1.25,0.25) -- (1.75,0.25); 
 \draw [line width=3pt, line cap=round] (1.25,0.25) -- (1.125,0.375); 
 \draw [line width=3pt, line cap=round] (1.75,1.25) -- (1.75,1.75); 
 \draw [line width=3pt, line cap=round] (1.75,1.75) -- (1.25,1.75); 
 \draw [line width=3pt, line cap=round] (1.25,1.25) -- (1.75,1.25); 
\end{tikzpicture} 

\begin{tikzpicture}[framed,background rectangle/.style={ultra thick,draw=black}]
 \draw [line width=3pt, line cap=round] (0.25,0.75) -- (0.25,1.25); 
 \draw [line width=3pt, line cap=round] (1,0.5) -- (0.75,0.25); 
 \draw [line width=3pt, line cap=round] (0.25,0.75) -- (0.25,0.25); 
 \draw [line width=3pt, line cap=round] (0.25,0.25) -- (0.75,0.25); 
 \draw [line width=3pt, line cap=round] (0.75,1.75) -- (0.25,1.75); 
 \draw [line width=3pt, line cap=round] (0.25,1.75) -- (0.25,1.25); 
 \draw [line width=3pt, line cap=round] (1,0.5) -- (0.75,0.25); 
 \draw [line width=3pt, line cap=round] (1.25,1.25) -- (0.75,1.25); 
 \draw [line width=3pt, line cap=round] (0.75,1.25) -- (0.75,0.75); 
 \draw [line width=3pt, line cap=round] (0.75,0.75) -- (0.875,0.625); 
 \draw [line width=3pt, line cap=round] (1,0.5) -- (1.25,0.75); 
 \draw [line width=3pt, line cap=round] (0.75,1.75) -- (1.25,1.75); 
 \draw [line width=3pt, line cap=round] (1.75,0.25) -- (1.75,0.75); 
 \draw [line width=3pt, line cap=round] (1.75,0.75) -- (1.25,0.75); 
 \draw [line width=3pt, line cap=round] (1,0.5) -- (1.25,0.75); 
 \draw [line width=3pt, line cap=round] (1.25,0.25) -- (1.75,0.25); 
 \draw [line width=3pt, line cap=round] (1.25,0.25) -- (1.125,0.375); 
 \draw [line width=3pt, line cap=round] (1.75,1.25) -- (1.75,1.75); 
 \draw [line width=3pt, line cap=round] (1.75,1.75) -- (1.25,1.75); 
 \draw [line width=3pt, line cap=round] (1.25,1.25) -- (1.75,1.25); 
\end{tikzpicture} 


\noindent
\begin{tikzpicture}[framed,background rectangle/.style={ultra thick,draw=black}]
 \draw [line width=3pt, line cap=round] (0.5,1) -- (0.25,1.25); 
 \draw [line width=3pt, line cap=round] (1,0.5) -- (0.75,0.25); 
 \draw [line width=3pt, line cap=round] (0.5,1) -- (0.75,0.75); 
 \draw [line width=3pt, line cap=round] (0.75,0.75) -- (0.875,0.625); 
 \draw [line width=3pt, line cap=round] (0.25,0.75) -- (0.25,0.25); 
 \draw [line width=3pt, line cap=round] (0.25,0.75) -- (0.375,0.875); 
 \draw [line width=3pt, line cap=round] (0.25,0.25) -- (0.75,0.25); 
 \draw [line width=3pt, line cap=round] (0.75,1.25) -- (0.75,1.75); 
 \draw [line width=3pt, line cap=round] (0.75,1.25) -- (0.625,1.125); 
 \draw [line width=3pt, line cap=round] (0.75,1.75) -- (0.25,1.75); 
 \draw [line width=3pt, line cap=round] (0.25,1.75) -- (0.25,1.25); 
 \draw [line width=3pt, line cap=round] (0.5,1) -- (0.25,1.25); 
 \draw [line width=3pt, line cap=round] (1,0.5) -- (0.75,0.25); 
 \draw [line width=3pt, line cap=round] (0.5,1) -- (0.75,0.75); 
 \draw [line width=3pt, line cap=round] (1,0.5) -- (1.25,0.75); 
 \draw [line width=3pt, line cap=round] (1.75,0.25) -- (1.75,0.75); 
 \draw [line width=3pt, line cap=round] (1.75,0.75) -- (1.25,0.75); 
 \draw [line width=3pt, line cap=round] (1,0.5) -- (1.25,0.75); 
 \draw [line width=3pt, line cap=round] (1.25,0.25) -- (1.75,0.25); 
 \draw [line width=3pt, line cap=round] (1.25,0.25) -- (1.125,0.375); 
 \draw [line width=3pt, line cap=round] (1.75,1.25) -- (1.75,1.75); 
 \draw [line width=3pt, line cap=round] (1.75,1.75) -- (1.25,1.75); 
 \draw [line width=3pt, line cap=round] (1.25,1.75) -- (1.25,1.25); 
 \draw [line width=3pt, line cap=round] (1.25,1.25) -- (1.75,1.25); 
\end{tikzpicture} 

\begin{tikzpicture}[framed,background rectangle/.style={ultra thick,draw=black}]
 \draw [line width=3pt, line cap=round] (1,0.5) -- (0.75,0.25); 
 \draw [line width=3pt, line cap=round] (0.75,0.75) -- (0.25,0.75); 
 \draw [line width=3pt, line cap=round] (0.75,0.75) -- (0.875,0.625); 
 \draw [line width=3pt, line cap=round] (0.25,0.75) -- (0.25,0.25); 
 \draw [line width=3pt, line cap=round] (0.25,0.25) -- (0.75,0.25); 
 \draw [line width=3pt, line cap=round] (0.75,1.25) -- (0.75,1.75); 
 \draw [line width=3pt, line cap=round] (0.75,1.75) -- (0.25,1.75); 
 \draw [line width=3pt, line cap=round] (0.25,1.75) -- (0.25,1.25); 
 \draw [line width=3pt, line cap=round] (0.25,1.25) -- (0.75,1.25); 
 \draw [line width=3pt, line cap=round] (1,0.5) -- (0.75,0.25); 
 \draw [line width=3pt, line cap=round] (1,0.5) -- (1.25,0.75); 
 \draw [line width=3pt, line cap=round] (1.75,0.25) -- (1.75,0.75); 
 \draw [line width=3pt, line cap=round] (1.75,0.75) -- (1.25,0.75); 
 \draw [line width=3pt, line cap=round] (1,0.5) -- (1.25,0.75); 
 \draw [line width=3pt, line cap=round] (1.25,0.25) -- (1.75,0.25); 
 \draw [line width=3pt, line cap=round] (1.25,0.25) -- (1.125,0.375); 
 \draw [line width=3pt, line cap=round] (1.75,1.25) -- (1.75,1.75); 
 \draw [line width=3pt, line cap=round] (1.75,1.75) -- (1.25,1.75); 
 \draw [line width=3pt, line cap=round] (1.25,1.75) -- (1.25,1.25); 
 \draw [line width=3pt, line cap=round] (1.25,1.25) -- (1.75,1.25); 
\end{tikzpicture} 

\begin{tikzpicture}[framed,background rectangle/.style={ultra thick,draw=black}]
 \draw [line width=3pt, line cap=round] (0.25,0.75) -- (0.25,1.25); 
 \draw [line width=3pt, line cap=round] (1,0.5) -- (0.75,0.25); 
 \draw [line width=3pt, line cap=round] (0.25,0.75) -- (0.25,0.25); 
 \draw [line width=3pt, line cap=round] (0.25,0.25) -- (0.75,0.25); 
 \draw [line width=3pt, line cap=round] (0.75,1.25) -- (0.75,1.75); 
 \draw [line width=3pt, line cap=round] (0.75,1.75) -- (0.25,1.75); 
 \draw [line width=3pt, line cap=round] (0.25,1.75) -- (0.25,1.25); 
 \draw [line width=3pt, line cap=round] (1,0.5) -- (0.75,0.25); 
 \draw [line width=3pt, line cap=round] (0.75,1.25) -- (0.75,0.75); 
 \draw [line width=3pt, line cap=round] (0.75,0.75) -- (0.875,0.625); 
 \draw [line width=3pt, line cap=round] (1,0.5) -- (1.25,0.75); 
 \draw [line width=3pt, line cap=round] (1.75,0.25) -- (1.75,0.75); 
 \draw [line width=3pt, line cap=round] (1.75,0.75) -- (1.25,0.75); 
 \draw [line width=3pt, line cap=round] (1,0.5) -- (1.25,0.75); 
 \draw [line width=3pt, line cap=round] (1.25,0.25) -- (1.75,0.25); 
 \draw [line width=3pt, line cap=round] (1.25,0.25) -- (1.125,0.375); 
 \draw [line width=3pt, line cap=round] (1.75,1.25) -- (1.75,1.75); 
 \draw [line width=3pt, line cap=round] (1.75,1.75) -- (1.25,1.75); 
 \draw [line width=3pt, line cap=round] (1.25,1.75) -- (1.25,1.25); 
 \draw [line width=3pt, line cap=round] (1.25,1.25) -- (1.75,1.25); 
\end{tikzpicture} 

\begin{tikzpicture}[framed,background rectangle/.style={ultra thick,draw=black}]
 \draw [line width=3pt, line cap=round] (0.5,1) -- (0.25,1.25); 
 \draw [line width=3pt, line cap=round] (1,0.5) -- (0.75,0.25); 
 \draw [line width=3pt, line cap=round] (0.5,1) -- (0.75,0.75); 
 \draw [line width=3pt, line cap=round] (0.75,0.75) -- (0.875,0.625); 
 \draw [line width=3pt, line cap=round] (0.25,0.75) -- (0.25,0.25); 
 \draw [line width=3pt, line cap=round] (0.25,0.75) -- (0.375,0.875); 
 \draw [line width=3pt, line cap=round] (0.25,0.25) -- (0.75,0.25); 
 \draw [line width=3pt, line cap=round] (1,1.5) -- (0.75,1.25); 
 \draw [line width=3pt, line cap=round] (0.75,1.25) -- (0.625,1.125); 
 \draw [line width=3pt, line cap=round] (0.75,1.75) -- (0.25,1.75); 
 \draw [line width=3pt, line cap=round] (0.75,1.75) -- (0.875,1.625); 
 \draw [line width=3pt, line cap=round] (0.25,1.75) -- (0.25,1.25); 
 \draw [line width=3pt, line cap=round] (0.5,1) -- (0.25,1.25); 
 \draw [line width=3pt, line cap=round] (1,0.5) -- (0.75,0.25); 
 \draw [line width=3pt, line cap=round] (1.25,0.75) -- (1.25,1.25); 
 \draw [line width=3pt, line cap=round] (1.25,1.25) -- (1.125,1.375); 
 \draw [line width=3pt, line cap=round] (1,1.5) -- (0.75,1.25); 
 \draw [line width=3pt, line cap=round] (0.5,1) -- (0.75,0.75); 
 \draw [line width=3pt, line cap=round] (1,0.5) -- (1.25,0.75); 
 \draw [line width=3pt, line cap=round] (1,1.5) -- (1.25,1.75); 
 \draw [line width=3pt, line cap=round] (1.75,0.25) -- (1.75,0.75); 
 \draw [line width=3pt, line cap=round] (1,0.5) -- (1.25,0.75); 
 \draw [line width=3pt, line cap=round] (1.25,0.25) -- (1.75,0.25); 
 \draw [line width=3pt, line cap=round] (1.25,0.25) -- (1.125,0.375); 
 \draw [line width=3pt, line cap=round] (1.75,1.25) -- (1.75,1.75); 
 \draw [line width=3pt, line cap=round] (1.75,1.75) -- (1.25,1.75); 
 \draw [line width=3pt, line cap=round] (1,1.5) -- (1.25,1.75); 
 \draw [line width=3pt, line cap=round] (1.75,1.25) -- (1.75,0.75); 
\end{tikzpicture} 

\begin{tikzpicture}[framed,background rectangle/.style={ultra thick,draw=black}]
 \draw [line width=3pt, line cap=round] (1,0.5) -- (0.75,0.25); 
 \draw [line width=3pt, line cap=round] (0.75,0.75) -- (0.25,0.75); 
 \draw [line width=3pt, line cap=round] (0.75,0.75) -- (0.875,0.625); 
 \draw [line width=3pt, line cap=round] (0.25,0.75) -- (0.25,0.25); 
 \draw [line width=3pt, line cap=round] (0.25,0.25) -- (0.75,0.25); 
 \draw [line width=3pt, line cap=round] (1,1.5) -- (0.75,1.25); 
 \draw [line width=3pt, line cap=round] (0.75,1.75) -- (0.25,1.75); 
 \draw [line width=3pt, line cap=round] (0.75,1.75) -- (0.875,1.625); 
 \draw [line width=3pt, line cap=round] (0.25,1.75) -- (0.25,1.25); 
 \draw [line width=3pt, line cap=round] (0.25,1.25) -- (0.75,1.25); 
 \draw [line width=3pt, line cap=round] (1,0.5) -- (0.75,0.25); 
 \draw [line width=3pt, line cap=round] (1.25,0.75) -- (1.25,1.25); 
 \draw [line width=3pt, line cap=round] (1.25,1.25) -- (1.125,1.375); 
 \draw [line width=3pt, line cap=round] (1,1.5) -- (0.75,1.25); 
 \draw [line width=3pt, line cap=round] (1,0.5) -- (1.25,0.75); 
 \draw [line width=3pt, line cap=round] (1,1.5) -- (1.25,1.75); 
 \draw [line width=3pt, line cap=round] (1.75,0.25) -- (1.75,0.75); 
 \draw [line width=3pt, line cap=round] (1,0.5) -- (1.25,0.75); 
 \draw [line width=3pt, line cap=round] (1.25,0.25) -- (1.75,0.25); 
 \draw [line width=3pt, line cap=round] (1.25,0.25) -- (1.125,0.375); 
 \draw [line width=3pt, line cap=round] (1.75,1.25) -- (1.75,1.75); 
 \draw [line width=3pt, line cap=round] (1.75,1.75) -- (1.25,1.75); 
 \draw [line width=3pt, line cap=round] (1,1.5) -- (1.25,1.75); 
 \draw [line width=3pt, line cap=round] (1.75,1.25) -- (1.75,0.75); 
\end{tikzpicture} 


\noindent
\begin{tikzpicture}[framed,background rectangle/.style={ultra thick,draw=black}]
 \draw [line width=3pt, line cap=round] (0.25,0.75) -- (0.25,1.25); 
 \draw [line width=3pt, line cap=round] (1,0.5) -- (0.75,0.25); 
 \draw [line width=3pt, line cap=round] (0.25,0.75) -- (0.25,0.25); 
 \draw [line width=3pt, line cap=round] (0.25,0.25) -- (0.75,0.25); 
 \draw [line width=3pt, line cap=round] (1,1.5) -- (0.75,1.25); 
 \draw [line width=3pt, line cap=round] (0.75,1.75) -- (0.25,1.75); 
 \draw [line width=3pt, line cap=round] (0.75,1.75) -- (0.875,1.625); 
 \draw [line width=3pt, line cap=round] (0.25,1.75) -- (0.25,1.25); 
 \draw [line width=3pt, line cap=round] (1,0.5) -- (0.75,0.25); 
 \draw [line width=3pt, line cap=round] (1.25,0.75) -- (1.25,1.25); 
 \draw [line width=3pt, line cap=round] (1.25,1.25) -- (1.125,1.375); 
 \draw [line width=3pt, line cap=round] (1,1.5) -- (0.75,1.25); 
 \draw [line width=3pt, line cap=round] (0.75,1.25) -- (0.75,0.75); 
 \draw [line width=3pt, line cap=round] (0.75,0.75) -- (0.875,0.625); 
 \draw [line width=3pt, line cap=round] (1,0.5) -- (1.25,0.75); 
 \draw [line width=3pt, line cap=round] (1,1.5) -- (1.25,1.75); 
 \draw [line width=3pt, line cap=round] (1.75,0.25) -- (1.75,0.75); 
 \draw [line width=3pt, line cap=round] (1,0.5) -- (1.25,0.75); 
 \draw [line width=3pt, line cap=round] (1.25,0.25) -- (1.75,0.25); 
 \draw [line width=3pt, line cap=round] (1.25,0.25) -- (1.125,0.375); 
 \draw [line width=3pt, line cap=round] (1.75,1.25) -- (1.75,1.75); 
 \draw [line width=3pt, line cap=round] (1.75,1.75) -- (1.25,1.75); 
 \draw [line width=3pt, line cap=round] (1,1.5) -- (1.25,1.75); 
 \draw [line width=3pt, line cap=round] (1.75,1.25) -- (1.75,0.75); 
\end{tikzpicture} 

\begin{tikzpicture}[framed,background rectangle/.style={ultra thick,draw=black}]
 \draw [line width=3pt, line cap=round] (0.5,1) -- (0.25,1.25); 
 \draw [line width=3pt, line cap=round] (1,0.5) -- (0.75,0.25); 
 \draw [line width=3pt, line cap=round] (0.5,1) -- (0.75,0.75); 
 \draw [line width=3pt, line cap=round] (0.75,0.75) -- (0.875,0.625); 
 \draw [line width=3pt, line cap=round] (0.25,0.75) -- (0.25,0.25); 
 \draw [line width=3pt, line cap=round] (0.25,0.75) -- (0.375,0.875); 
 \draw [line width=3pt, line cap=round] (0.25,0.25) -- (0.75,0.25); 
 \draw [line width=3pt, line cap=round] (0.75,1.75) -- (0.25,1.75); 
 \draw [line width=3pt, line cap=round] (0.25,1.75) -- (0.25,1.25); 
 \draw [line width=3pt, line cap=round] (0.5,1) -- (0.25,1.25); 
 \draw [line width=3pt, line cap=round] (1,0.5) -- (0.75,0.25); 
 \draw [line width=3pt, line cap=round] (1.25,0.75) -- (1.25,1.25); 
 \draw [line width=3pt, line cap=round] (1.25,1.25) -- (0.75,1.25); 
 \draw [line width=3pt, line cap=round] (0.75,1.25) -- (0.625,1.125); 
 \draw [line width=3pt, line cap=round] (0.5,1) -- (0.75,0.75); 
 \draw [line width=3pt, line cap=round] (1,0.5) -- (1.25,0.75); 
 \draw [line width=3pt, line cap=round] (0.75,1.75) -- (1.25,1.75); 
 \draw [line width=3pt, line cap=round] (1.75,0.25) -- (1.75,0.75); 
 \draw [line width=3pt, line cap=round] (1,0.5) -- (1.25,0.75); 
 \draw [line width=3pt, line cap=round] (1.25,0.25) -- (1.75,0.25); 
 \draw [line width=3pt, line cap=round] (1.25,0.25) -- (1.125,0.375); 
 \draw [line width=3pt, line cap=round] (1.75,1.25) -- (1.75,1.75); 
 \draw [line width=3pt, line cap=round] (1.75,1.75) -- (1.25,1.75); 
 \draw [line width=3pt, line cap=round] (1.75,1.25) -- (1.75,0.75); 
\end{tikzpicture} 

\begin{tikzpicture}[framed,background rectangle/.style={ultra thick,draw=black}]
 \draw [line width=3pt, line cap=round] (1,0.5) -- (0.75,0.25); 
 \draw [line width=3pt, line cap=round] (0.75,0.75) -- (0.25,0.75); 
 \draw [line width=3pt, line cap=round] (0.75,0.75) -- (0.875,0.625); 
 \draw [line width=3pt, line cap=round] (0.25,0.75) -- (0.25,0.25); 
 \draw [line width=3pt, line cap=round] (0.25,0.25) -- (0.75,0.25); 
 \draw [line width=3pt, line cap=round] (0.75,1.75) -- (0.25,1.75); 
 \draw [line width=3pt, line cap=round] (0.25,1.75) -- (0.25,1.25); 
 \draw [line width=3pt, line cap=round] (0.25,1.25) -- (0.75,1.25); 
 \draw [line width=3pt, line cap=round] (1,0.5) -- (0.75,0.25); 
 \draw [line width=3pt, line cap=round] (1.25,0.75) -- (1.25,1.25); 
 \draw [line width=3pt, line cap=round] (1.25,1.25) -- (0.75,1.25); 
 \draw [line width=3pt, line cap=round] (1,0.5) -- (1.25,0.75); 
 \draw [line width=3pt, line cap=round] (0.75,1.75) -- (1.25,1.75); 
 \draw [line width=3pt, line cap=round] (1.75,0.25) -- (1.75,0.75); 
 \draw [line width=3pt, line cap=round] (1,0.5) -- (1.25,0.75); 
 \draw [line width=3pt, line cap=round] (1.25,0.25) -- (1.75,0.25); 
 \draw [line width=3pt, line cap=round] (1.25,0.25) -- (1.125,0.375); 
 \draw [line width=3pt, line cap=round] (1.75,1.25) -- (1.75,1.75); 
 \draw [line width=3pt, line cap=round] (1.75,1.75) -- (1.25,1.75); 
 \draw [line width=3pt, line cap=round] (1.75,1.25) -- (1.75,0.75); 
\end{tikzpicture} 

\begin{tikzpicture}[framed,background rectangle/.style={ultra thick,draw=black}]
 \draw [line width=3pt, line cap=round] (0.25,0.75) -- (0.25,1.25); 
 \draw [line width=3pt, line cap=round] (1,0.5) -- (0.75,0.25); 
 \draw [line width=3pt, line cap=round] (0.25,0.75) -- (0.25,0.25); 
 \draw [line width=3pt, line cap=round] (0.25,0.25) -- (0.75,0.25); 
 \draw [line width=3pt, line cap=round] (0.75,1.75) -- (0.25,1.75); 
 \draw [line width=3pt, line cap=round] (0.25,1.75) -- (0.25,1.25); 
 \draw [line width=3pt, line cap=round] (1,0.5) -- (0.75,0.25); 
 \draw [line width=3pt, line cap=round] (1.25,0.75) -- (1.25,1.25); 
 \draw [line width=3pt, line cap=round] (1.25,1.25) -- (0.75,1.25); 
 \draw [line width=3pt, line cap=round] (0.75,1.25) -- (0.75,0.75); 
 \draw [line width=3pt, line cap=round] (0.75,0.75) -- (0.875,0.625); 
 \draw [line width=3pt, line cap=round] (1,0.5) -- (1.25,0.75); 
 \draw [line width=3pt, line cap=round] (0.75,1.75) -- (1.25,1.75); 
 \draw [line width=3pt, line cap=round] (1.75,0.25) -- (1.75,0.75); 
 \draw [line width=3pt, line cap=round] (1,0.5) -- (1.25,0.75); 
 \draw [line width=3pt, line cap=round] (1.25,0.25) -- (1.75,0.25); 
 \draw [line width=3pt, line cap=round] (1.25,0.25) -- (1.125,0.375); 
 \draw [line width=3pt, line cap=round] (1.75,1.25) -- (1.75,1.75); 
 \draw [line width=3pt, line cap=round] (1.75,1.75) -- (1.25,1.75); 
 \draw [line width=3pt, line cap=round] (1.75,1.25) -- (1.75,0.75); 
\end{tikzpicture} 

\begin{tikzpicture}[framed,background rectangle/.style={ultra thick,draw=black}]
 \draw [line width=3pt, line cap=round] (0.5,1) -- (0.25,1.25); 
 \draw [line width=3pt, line cap=round] (1,0.5) -- (0.75,0.25); 
 \draw [line width=3pt, line cap=round] (0.5,1) -- (0.75,0.75); 
 \draw [line width=3pt, line cap=round] (0.75,0.75) -- (0.875,0.625); 
 \draw [line width=3pt, line cap=round] (0.25,0.75) -- (0.25,0.25); 
 \draw [line width=3pt, line cap=round] (0.25,0.75) -- (0.375,0.875); 
 \draw [line width=3pt, line cap=round] (0.25,0.25) -- (0.75,0.25); 
 \draw [line width=3pt, line cap=round] (0.75,1.25) -- (0.75,1.75); 
 \draw [line width=3pt, line cap=round] (0.75,1.25) -- (0.625,1.125); 
 \draw [line width=3pt, line cap=round] (0.75,1.75) -- (0.25,1.75); 
 \draw [line width=3pt, line cap=round] (0.25,1.75) -- (0.25,1.25); 
 \draw [line width=3pt, line cap=round] (0.5,1) -- (0.25,1.25); 
 \draw [line width=3pt, line cap=round] (1,0.5) -- (0.75,0.25); 
 \draw [line width=3pt, line cap=round] (1.25,0.75) -- (1.25,1.25); 
 \draw [line width=3pt, line cap=round] (0.5,1) -- (0.75,0.75); 
 \draw [line width=3pt, line cap=round] (1,0.5) -- (1.25,0.75); 
 \draw [line width=3pt, line cap=round] (1.75,0.25) -- (1.75,0.75); 
 \draw [line width=3pt, line cap=round] (1,0.5) -- (1.25,0.75); 
 \draw [line width=3pt, line cap=round] (1.25,0.25) -- (1.75,0.25); 
 \draw [line width=3pt, line cap=round] (1.25,0.25) -- (1.125,0.375); 
 \draw [line width=3pt, line cap=round] (1.75,1.25) -- (1.75,1.75); 
 \draw [line width=3pt, line cap=round] (1.75,1.75) -- (1.25,1.75); 
 \draw [line width=3pt, line cap=round] (1.25,1.75) -- (1.25,1.25); 
 \draw [line width=3pt, line cap=round] (1.75,1.25) -- (1.75,0.75); 
\end{tikzpicture} 


\noindent
\begin{tikzpicture}[framed,background rectangle/.style={ultra thick,draw=black}]
 \draw [line width=3pt, line cap=round] (1,0.5) -- (0.75,0.25); 
 \draw [line width=3pt, line cap=round] (0.75,0.75) -- (0.25,0.75); 
 \draw [line width=3pt, line cap=round] (0.75,0.75) -- (0.875,0.625); 
 \draw [line width=3pt, line cap=round] (0.25,0.75) -- (0.25,0.25); 
 \draw [line width=3pt, line cap=round] (0.25,0.25) -- (0.75,0.25); 
 \draw [line width=3pt, line cap=round] (0.75,1.25) -- (0.75,1.75); 
 \draw [line width=3pt, line cap=round] (0.75,1.75) -- (0.25,1.75); 
 \draw [line width=3pt, line cap=round] (0.25,1.75) -- (0.25,1.25); 
 \draw [line width=3pt, line cap=round] (0.25,1.25) -- (0.75,1.25); 
 \draw [line width=3pt, line cap=round] (1,0.5) -- (0.75,0.25); 
 \draw [line width=3pt, line cap=round] (1.25,0.75) -- (1.25,1.25); 
 \draw [line width=3pt, line cap=round] (1,0.5) -- (1.25,0.75); 
 \draw [line width=3pt, line cap=round] (1.75,0.25) -- (1.75,0.75); 
 \draw [line width=3pt, line cap=round] (1,0.5) -- (1.25,0.75); 
 \draw [line width=3pt, line cap=round] (1.25,0.25) -- (1.75,0.25); 
 \draw [line width=3pt, line cap=round] (1.25,0.25) -- (1.125,0.375); 
 \draw [line width=3pt, line cap=round] (1.75,1.25) -- (1.75,1.75); 
 \draw [line width=3pt, line cap=round] (1.75,1.75) -- (1.25,1.75); 
 \draw [line width=3pt, line cap=round] (1.25,1.75) -- (1.25,1.25); 
 \draw [line width=3pt, line cap=round] (1.75,1.25) -- (1.75,0.75); 
\end{tikzpicture} 

\begin{tikzpicture}[framed,background rectangle/.style={ultra thick,draw=black}]
 \draw [line width=3pt, line cap=round] (0.25,0.75) -- (0.25,1.25); 
 \draw [line width=3pt, line cap=round] (1,0.5) -- (0.75,0.25); 
 \draw [line width=3pt, line cap=round] (0.25,0.75) -- (0.25,0.25); 
 \draw [line width=3pt, line cap=round] (0.25,0.25) -- (0.75,0.25); 
 \draw [line width=3pt, line cap=round] (0.75,1.25) -- (0.75,1.75); 
 \draw [line width=3pt, line cap=round] (0.75,1.75) -- (0.25,1.75); 
 \draw [line width=3pt, line cap=round] (0.25,1.75) -- (0.25,1.25); 
 \draw [line width=3pt, line cap=round] (1,0.5) -- (0.75,0.25); 
 \draw [line width=3pt, line cap=round] (1.25,0.75) -- (1.25,1.25); 
 \draw [line width=3pt, line cap=round] (0.75,1.25) -- (0.75,0.75); 
 \draw [line width=3pt, line cap=round] (0.75,0.75) -- (0.875,0.625); 
 \draw [line width=3pt, line cap=round] (1,0.5) -- (1.25,0.75); 
 \draw [line width=3pt, line cap=round] (1.75,0.25) -- (1.75,0.75); 
 \draw [line width=3pt, line cap=round] (1,0.5) -- (1.25,0.75); 
 \draw [line width=3pt, line cap=round] (1.25,0.25) -- (1.75,0.25); 
 \draw [line width=3pt, line cap=round] (1.25,0.25) -- (1.125,0.375); 
 \draw [line width=3pt, line cap=round] (1.75,1.25) -- (1.75,1.75); 
 \draw [line width=3pt, line cap=round] (1.75,1.75) -- (1.25,1.75); 
 \draw [line width=3pt, line cap=round] (1.25,1.75) -- (1.25,1.25); 
 \draw [line width=3pt, line cap=round] (1.75,1.25) -- (1.75,0.75); 
\end{tikzpicture} 

\begin{tikzpicture}[framed,background rectangle/.style={ultra thick,draw=black}]
 \draw [line width=3pt, line cap=round] (0.5,1) -- (0.25,1.25); 
 \draw [line width=3pt, line cap=round] (0.5,1) -- (0.75,0.75); 
 \draw [line width=3pt, line cap=round] (0.25,0.75) -- (0.25,0.25); 
 \draw [line width=3pt, line cap=round] (0.25,0.75) -- (0.375,0.875); 
 \draw [line width=3pt, line cap=round] (0.25,0.25) -- (0.75,0.25); 
 \draw [line width=3pt, line cap=round] (1,1.5) -- (0.75,1.25); 
 \draw [line width=3pt, line cap=round] (0.75,1.25) -- (0.625,1.125); 
 \draw [line width=3pt, line cap=round] (0.75,1.75) -- (0.25,1.75); 
 \draw [line width=3pt, line cap=round] (0.75,1.75) -- (0.875,1.625); 
 \draw [line width=3pt, line cap=round] (0.25,1.75) -- (0.25,1.25); 
 \draw [line width=3pt, line cap=round] (0.5,1) -- (0.25,1.25); 
 \draw [line width=3pt, line cap=round] (1.25,0.25) -- (0.75,0.25); 
 \draw [line width=3pt, line cap=round] (1.5,1) -- (1.25,1.25); 
 \draw [line width=3pt, line cap=round] (1,1.5) -- (0.75,1.25); 
 \draw [line width=3pt, line cap=round] (0.5,1) -- (0.75,0.75); 
 \draw [line width=3pt, line cap=round] (0.75,0.75) -- (1.25,0.75); 
 \draw [line width=3pt, line cap=round] (1.25,0.75) -- (1.375,0.875); 
 \draw [line width=3pt, line cap=round] (1,1.5) -- (1.25,1.75); 
 \draw [line width=3pt, line cap=round] (1.75,0.25) -- (1.75,0.75); 
 \draw [line width=3pt, line cap=round] (1.5,1) -- (1.75,0.75); 
 \draw [line width=3pt, line cap=round] (1.25,0.25) -- (1.75,0.25); 
 \draw [line width=3pt, line cap=round] (1.75,1.25) -- (1.75,1.75); 
 \draw [line width=3pt, line cap=round] (1.75,1.25) -- (1.625,1.125); 
 \draw [line width=3pt, line cap=round] (1.75,1.75) -- (1.25,1.75); 
 \draw [line width=3pt, line cap=round] (1,1.5) -- (1.25,1.75); 
 \draw [line width=3pt, line cap=round] (1.5,1) -- (1.25,1.25); 
 \draw [line width=3pt, line cap=round] (1.25,1.25) -- (1.125,1.375); 
 \draw [line width=3pt, line cap=round] (1.5,1) -- (1.75,0.75); 
\end{tikzpicture} 

\begin{tikzpicture}[framed,background rectangle/.style={ultra thick,draw=black}]
 \draw [line width=3pt, line cap=round] (0.75,0.75) -- (0.25,0.75); 
 \draw [line width=3pt, line cap=round] (0.25,0.75) -- (0.25,0.25); 
 \draw [line width=3pt, line cap=round] (0.25,0.25) -- (0.75,0.25); 
 \draw [line width=3pt, line cap=round] (1,1.5) -- (0.75,1.25); 
 \draw [line width=3pt, line cap=round] (0.75,1.75) -- (0.25,1.75); 
 \draw [line width=3pt, line cap=round] (0.75,1.75) -- (0.875,1.625); 
 \draw [line width=3pt, line cap=round] (0.25,1.75) -- (0.25,1.25); 
 \draw [line width=3pt, line cap=round] (0.25,1.25) -- (0.75,1.25); 
 \draw [line width=3pt, line cap=round] (1.25,0.25) -- (0.75,0.25); 
 \draw [line width=3pt, line cap=round] (1.5,1) -- (1.25,1.25); 
 \draw [line width=3pt, line cap=round] (1,1.5) -- (0.75,1.25); 
 \draw [line width=3pt, line cap=round] (0.75,0.75) -- (1.25,0.75); 
 \draw [line width=3pt, line cap=round] (1.25,0.75) -- (1.375,0.875); 
 \draw [line width=3pt, line cap=round] (1,1.5) -- (1.25,1.75); 
 \draw [line width=3pt, line cap=round] (1.75,0.25) -- (1.75,0.75); 
 \draw [line width=3pt, line cap=round] (1.5,1) -- (1.75,0.75); 
 \draw [line width=3pt, line cap=round] (1.25,0.25) -- (1.75,0.25); 
 \draw [line width=3pt, line cap=round] (1.75,1.25) -- (1.75,1.75); 
 \draw [line width=3pt, line cap=round] (1.75,1.25) -- (1.625,1.125); 
 \draw [line width=3pt, line cap=round] (1.75,1.75) -- (1.25,1.75); 
 \draw [line width=3pt, line cap=round] (1,1.5) -- (1.25,1.75); 
 \draw [line width=3pt, line cap=round] (1.5,1) -- (1.25,1.25); 
 \draw [line width=3pt, line cap=round] (1.25,1.25) -- (1.125,1.375); 
 \draw [line width=3pt, line cap=round] (1.5,1) -- (1.75,0.75); 
\end{tikzpicture} 

\begin{tikzpicture}[framed,background rectangle/.style={ultra thick,draw=black}]
 \draw [line width=3pt, line cap=round] (0.25,0.75) -- (0.25,1.25); 
 \draw [line width=3pt, line cap=round] (0.25,0.75) -- (0.25,0.25); 
 \draw [line width=3pt, line cap=round] (0.25,0.25) -- (0.75,0.25); 
 \draw [line width=3pt, line cap=round] (1,1.5) -- (0.75,1.25); 
 \draw [line width=3pt, line cap=round] (0.75,1.75) -- (0.25,1.75); 
 \draw [line width=3pt, line cap=round] (0.75,1.75) -- (0.875,1.625); 
 \draw [line width=3pt, line cap=round] (0.25,1.75) -- (0.25,1.25); 
 \draw [line width=3pt, line cap=round] (1.25,0.25) -- (0.75,0.25); 
 \draw [line width=3pt, line cap=round] (1.5,1) -- (1.25,1.25); 
 \draw [line width=3pt, line cap=round] (1,1.5) -- (0.75,1.25); 
 \draw [line width=3pt, line cap=round] (0.75,1.25) -- (0.75,0.75); 
 \draw [line width=3pt, line cap=round] (0.75,0.75) -- (1.25,0.75); 
 \draw [line width=3pt, line cap=round] (1.25,0.75) -- (1.375,0.875); 
 \draw [line width=3pt, line cap=round] (1,1.5) -- (1.25,1.75); 
 \draw [line width=3pt, line cap=round] (1.75,0.25) -- (1.75,0.75); 
 \draw [line width=3pt, line cap=round] (1.5,1) -- (1.75,0.75); 
 \draw [line width=3pt, line cap=round] (1.25,0.25) -- (1.75,0.25); 
 \draw [line width=3pt, line cap=round] (1.75,1.25) -- (1.75,1.75); 
 \draw [line width=3pt, line cap=round] (1.75,1.25) -- (1.625,1.125); 
 \draw [line width=3pt, line cap=round] (1.75,1.75) -- (1.25,1.75); 
 \draw [line width=3pt, line cap=round] (1,1.5) -- (1.25,1.75); 
 \draw [line width=3pt, line cap=round] (1.5,1) -- (1.25,1.25); 
 \draw [line width=3pt, line cap=round] (1.25,1.25) -- (1.125,1.375); 
 \draw [line width=3pt, line cap=round] (1.5,1) -- (1.75,0.75); 
\end{tikzpicture} 


\noindent
\begin{tikzpicture}[framed,background rectangle/.style={ultra thick,draw=black}]
 \draw [line width=3pt, line cap=round] (0.5,1) -- (0.25,1.25); 
 \draw [line width=3pt, line cap=round] (0.5,1) -- (0.75,0.75); 
 \draw [line width=3pt, line cap=round] (0.25,0.75) -- (0.25,0.25); 
 \draw [line width=3pt, line cap=round] (0.25,0.75) -- (0.375,0.875); 
 \draw [line width=3pt, line cap=round] (0.25,0.25) -- (0.75,0.25); 
 \draw [line width=3pt, line cap=round] (0.75,1.75) -- (0.25,1.75); 
 \draw [line width=3pt, line cap=round] (0.25,1.75) -- (0.25,1.25); 
 \draw [line width=3pt, line cap=round] (0.5,1) -- (0.25,1.25); 
 \draw [line width=3pt, line cap=round] (1.25,0.25) -- (0.75,0.25); 
 \draw [line width=3pt, line cap=round] (1.5,1) -- (1.25,1.25); 
 \draw [line width=3pt, line cap=round] (1.25,1.25) -- (0.75,1.25); 
 \draw [line width=3pt, line cap=round] (0.75,1.25) -- (0.625,1.125); 
 \draw [line width=3pt, line cap=round] (0.5,1) -- (0.75,0.75); 
 \draw [line width=3pt, line cap=round] (0.75,0.75) -- (1.25,0.75); 
 \draw [line width=3pt, line cap=round] (1.25,0.75) -- (1.375,0.875); 
 \draw [line width=3pt, line cap=round] (0.75,1.75) -- (1.25,1.75); 
 \draw [line width=3pt, line cap=round] (1.75,0.25) -- (1.75,0.75); 
 \draw [line width=3pt, line cap=round] (1.5,1) -- (1.75,0.75); 
 \draw [line width=3pt, line cap=round] (1.25,0.25) -- (1.75,0.25); 
 \draw [line width=3pt, line cap=round] (1.75,1.25) -- (1.75,1.75); 
 \draw [line width=3pt, line cap=round] (1.75,1.25) -- (1.625,1.125); 
 \draw [line width=3pt, line cap=round] (1.75,1.75) -- (1.25,1.75); 
 \draw [line width=3pt, line cap=round] (1.5,1) -- (1.25,1.25); 
 \draw [line width=3pt, line cap=round] (1.5,1) -- (1.75,0.75); 
\end{tikzpicture} 

\begin{tikzpicture}[framed,background rectangle/.style={ultra thick,draw=black}]
 \draw [line width=3pt, line cap=round] (0.75,0.75) -- (0.25,0.75); 
 \draw [line width=3pt, line cap=round] (0.25,0.75) -- (0.25,0.25); 
 \draw [line width=3pt, line cap=round] (0.25,0.25) -- (0.75,0.25); 
 \draw [line width=3pt, line cap=round] (0.75,1.75) -- (0.25,1.75); 
 \draw [line width=3pt, line cap=round] (0.25,1.75) -- (0.25,1.25); 
 \draw [line width=3pt, line cap=round] (0.25,1.25) -- (0.75,1.25); 
 \draw [line width=3pt, line cap=round] (1.25,0.25) -- (0.75,0.25); 
 \draw [line width=3pt, line cap=round] (1.5,1) -- (1.25,1.25); 
 \draw [line width=3pt, line cap=round] (1.25,1.25) -- (0.75,1.25); 
 \draw [line width=3pt, line cap=round] (0.75,0.75) -- (1.25,0.75); 
 \draw [line width=3pt, line cap=round] (1.25,0.75) -- (1.375,0.875); 
 \draw [line width=3pt, line cap=round] (0.75,1.75) -- (1.25,1.75); 
 \draw [line width=3pt, line cap=round] (1.75,0.25) -- (1.75,0.75); 
 \draw [line width=3pt, line cap=round] (1.5,1) -- (1.75,0.75); 
 \draw [line width=3pt, line cap=round] (1.25,0.25) -- (1.75,0.25); 
 \draw [line width=3pt, line cap=round] (1.75,1.25) -- (1.75,1.75); 
 \draw [line width=3pt, line cap=round] (1.75,1.25) -- (1.625,1.125); 
 \draw [line width=3pt, line cap=round] (1.75,1.75) -- (1.25,1.75); 
 \draw [line width=3pt, line cap=round] (1.5,1) -- (1.25,1.25); 
 \draw [line width=3pt, line cap=round] (1.5,1) -- (1.75,0.75); 
\end{tikzpicture} 

\begin{tikzpicture}[framed,background rectangle/.style={ultra thick,draw=black}]
 \draw [line width=3pt, line cap=round] (0.25,0.75) -- (0.25,1.25); 
 \draw [line width=3pt, line cap=round] (0.25,0.75) -- (0.25,0.25); 
 \draw [line width=3pt, line cap=round] (0.25,0.25) -- (0.75,0.25); 
 \draw [line width=3pt, line cap=round] (0.75,1.75) -- (0.25,1.75); 
 \draw [line width=3pt, line cap=round] (0.25,1.75) -- (0.25,1.25); 
 \draw [line width=3pt, line cap=round] (1.25,0.25) -- (0.75,0.25); 
 \draw [line width=3pt, line cap=round] (1.5,1) -- (1.25,1.25); 
 \draw [line width=3pt, line cap=round] (1.25,1.25) -- (0.75,1.25); 
 \draw [line width=3pt, line cap=round] (0.75,1.25) -- (0.75,0.75); 
 \draw [line width=3pt, line cap=round] (0.75,0.75) -- (1.25,0.75); 
 \draw [line width=3pt, line cap=round] (1.25,0.75) -- (1.375,0.875); 
 \draw [line width=3pt, line cap=round] (0.75,1.75) -- (1.25,1.75); 
 \draw [line width=3pt, line cap=round] (1.75,0.25) -- (1.75,0.75); 
 \draw [line width=3pt, line cap=round] (1.5,1) -- (1.75,0.75); 
 \draw [line width=3pt, line cap=round] (1.25,0.25) -- (1.75,0.25); 
 \draw [line width=3pt, line cap=round] (1.75,1.25) -- (1.75,1.75); 
 \draw [line width=3pt, line cap=round] (1.75,1.25) -- (1.625,1.125); 
 \draw [line width=3pt, line cap=round] (1.75,1.75) -- (1.25,1.75); 
 \draw [line width=3pt, line cap=round] (1.5,1) -- (1.25,1.25); 
 \draw [line width=3pt, line cap=round] (1.5,1) -- (1.75,0.75); 
\end{tikzpicture} 

\begin{tikzpicture}[framed,background rectangle/.style={ultra thick,draw=black}]
 \draw [line width=3pt, line cap=round] (0.5,1) -- (0.25,1.25); 
 \draw [line width=3pt, line cap=round] (0.5,1) -- (0.75,0.75); 
 \draw [line width=3pt, line cap=round] (0.25,0.75) -- (0.25,0.25); 
 \draw [line width=3pt, line cap=round] (0.25,0.75) -- (0.375,0.875); 
 \draw [line width=3pt, line cap=round] (0.25,0.25) -- (0.75,0.25); 
 \draw [line width=3pt, line cap=round] (0.75,1.25) -- (0.75,1.75); 
 \draw [line width=3pt, line cap=round] (0.75,1.25) -- (0.625,1.125); 
 \draw [line width=3pt, line cap=round] (0.75,1.75) -- (0.25,1.75); 
 \draw [line width=3pt, line cap=round] (0.25,1.75) -- (0.25,1.25); 
 \draw [line width=3pt, line cap=round] (0.5,1) -- (0.25,1.25); 
 \draw [line width=3pt, line cap=round] (1.25,0.25) -- (0.75,0.25); 
 \draw [line width=3pt, line cap=round] (1.5,1) -- (1.25,1.25); 
 \draw [line width=3pt, line cap=round] (0.5,1) -- (0.75,0.75); 
 \draw [line width=3pt, line cap=round] (0.75,0.75) -- (1.25,0.75); 
 \draw [line width=3pt, line cap=round] (1.25,0.75) -- (1.375,0.875); 
 \draw [line width=3pt, line cap=round] (1.75,0.25) -- (1.75,0.75); 
 \draw [line width=3pt, line cap=round] (1.5,1) -- (1.75,0.75); 
 \draw [line width=3pt, line cap=round] (1.25,0.25) -- (1.75,0.25); 
 \draw [line width=3pt, line cap=round] (1.75,1.25) -- (1.75,1.75); 
 \draw [line width=3pt, line cap=round] (1.75,1.25) -- (1.625,1.125); 
 \draw [line width=3pt, line cap=round] (1.75,1.75) -- (1.25,1.75); 
 \draw [line width=3pt, line cap=round] (1.25,1.75) -- (1.25,1.25); 
 \draw [line width=3pt, line cap=round] (1.5,1) -- (1.25,1.25); 
 \draw [line width=3pt, line cap=round] (1.5,1) -- (1.75,0.75); 
\end{tikzpicture} 

\begin{tikzpicture}[framed,background rectangle/.style={ultra thick,draw=black}]
 \draw [line width=3pt, line cap=round] (0.75,0.75) -- (0.25,0.75); 
 \draw [line width=3pt, line cap=round] (0.25,0.75) -- (0.25,0.25); 
 \draw [line width=3pt, line cap=round] (0.25,0.25) -- (0.75,0.25); 
 \draw [line width=3pt, line cap=round] (0.75,1.25) -- (0.75,1.75); 
 \draw [line width=3pt, line cap=round] (0.75,1.75) -- (0.25,1.75); 
 \draw [line width=3pt, line cap=round] (0.25,1.75) -- (0.25,1.25); 
 \draw [line width=3pt, line cap=round] (0.25,1.25) -- (0.75,1.25); 
 \draw [line width=3pt, line cap=round] (1.25,0.25) -- (0.75,0.25); 
 \draw [line width=3pt, line cap=round] (1.5,1) -- (1.25,1.25); 
 \draw [line width=3pt, line cap=round] (0.75,0.75) -- (1.25,0.75); 
 \draw [line width=3pt, line cap=round] (1.25,0.75) -- (1.375,0.875); 
 \draw [line width=3pt, line cap=round] (1.75,0.25) -- (1.75,0.75); 
 \draw [line width=3pt, line cap=round] (1.5,1) -- (1.75,0.75); 
 \draw [line width=3pt, line cap=round] (1.25,0.25) -- (1.75,0.25); 
 \draw [line width=3pt, line cap=round] (1.75,1.25) -- (1.75,1.75); 
 \draw [line width=3pt, line cap=round] (1.75,1.25) -- (1.625,1.125); 
 \draw [line width=3pt, line cap=round] (1.75,1.75) -- (1.25,1.75); 
 \draw [line width=3pt, line cap=round] (1.25,1.75) -- (1.25,1.25); 
 \draw [line width=3pt, line cap=round] (1.5,1) -- (1.25,1.25); 
 \draw [line width=3pt, line cap=round] (1.5,1) -- (1.75,0.75); 
\end{tikzpicture} 


\noindent
\begin{tikzpicture}[framed,background rectangle/.style={ultra thick,draw=black}]
 \draw [line width=3pt, line cap=round] (0.25,0.75) -- (0.25,1.25); 
 \draw [line width=3pt, line cap=round] (0.25,0.75) -- (0.25,0.25); 
 \draw [line width=3pt, line cap=round] (0.25,0.25) -- (0.75,0.25); 
 \draw [line width=3pt, line cap=round] (0.75,1.25) -- (0.75,1.75); 
 \draw [line width=3pt, line cap=round] (0.75,1.75) -- (0.25,1.75); 
 \draw [line width=3pt, line cap=round] (0.25,1.75) -- (0.25,1.25); 
 \draw [line width=3pt, line cap=round] (1.25,0.25) -- (0.75,0.25); 
 \draw [line width=3pt, line cap=round] (1.5,1) -- (1.25,1.25); 
 \draw [line width=3pt, line cap=round] (0.75,1.25) -- (0.75,0.75); 
 \draw [line width=3pt, line cap=round] (0.75,0.75) -- (1.25,0.75); 
 \draw [line width=3pt, line cap=round] (1.25,0.75) -- (1.375,0.875); 
 \draw [line width=3pt, line cap=round] (1.75,0.25) -- (1.75,0.75); 
 \draw [line width=3pt, line cap=round] (1.5,1) -- (1.75,0.75); 
 \draw [line width=3pt, line cap=round] (1.25,0.25) -- (1.75,0.25); 
 \draw [line width=3pt, line cap=round] (1.75,1.25) -- (1.75,1.75); 
 \draw [line width=3pt, line cap=round] (1.75,1.25) -- (1.625,1.125); 
 \draw [line width=3pt, line cap=round] (1.75,1.75) -- (1.25,1.75); 
 \draw [line width=3pt, line cap=round] (1.25,1.75) -- (1.25,1.25); 
 \draw [line width=3pt, line cap=round] (1.5,1) -- (1.25,1.25); 
 \draw [line width=3pt, line cap=round] (1.5,1) -- (1.75,0.75); 
\end{tikzpicture} 

\begin{tikzpicture}[framed,background rectangle/.style={ultra thick,draw=black}]
 \draw [line width=3pt, line cap=round] (0.5,1) -- (0.25,1.25); 
 \draw [line width=3pt, line cap=round] (0.5,1) -- (0.75,0.75); 
 \draw [line width=3pt, line cap=round] (0.25,0.75) -- (0.25,0.25); 
 \draw [line width=3pt, line cap=round] (0.25,0.75) -- (0.375,0.875); 
 \draw [line width=3pt, line cap=round] (0.25,0.25) -- (0.75,0.25); 
 \draw [line width=3pt, line cap=round] (1,1.5) -- (0.75,1.25); 
 \draw [line width=3pt, line cap=round] (0.75,1.25) -- (0.625,1.125); 
 \draw [line width=3pt, line cap=round] (0.75,1.75) -- (0.25,1.75); 
 \draw [line width=3pt, line cap=round] (0.75,1.75) -- (0.875,1.625); 
 \draw [line width=3pt, line cap=round] (0.25,1.75) -- (0.25,1.25); 
 \draw [line width=3pt, line cap=round] (0.5,1) -- (0.25,1.25); 
 \draw [line width=3pt, line cap=round] (1.25,0.25) -- (0.75,0.25); 
 \draw [line width=3pt, line cap=round] (1,1.5) -- (0.75,1.25); 
 \draw [line width=3pt, line cap=round] (0.5,1) -- (0.75,0.75); 
 \draw [line width=3pt, line cap=round] (0.75,0.75) -- (1.25,0.75); 
 \draw [line width=3pt, line cap=round] (1,1.5) -- (1.25,1.75); 
 \draw [line width=3pt, line cap=round] (1.75,0.25) -- (1.75,0.75); 
 \draw [line width=3pt, line cap=round] (1.75,0.75) -- (1.25,0.75); 
 \draw [line width=3pt, line cap=round] (1.25,0.25) -- (1.75,0.25); 
 \draw [line width=3pt, line cap=round] (1.75,1.25) -- (1.75,1.75); 
 \draw [line width=3pt, line cap=round] (1.75,1.75) -- (1.25,1.75); 
 \draw [line width=3pt, line cap=round] (1,1.5) -- (1.25,1.75); 
 \draw [line width=3pt, line cap=round] (1.25,1.25) -- (1.75,1.25); 
 \draw [line width=3pt, line cap=round] (1.25,1.25) -- (1.125,1.375); 
\end{tikzpicture} 

\begin{tikzpicture}[framed,background rectangle/.style={ultra thick,draw=black}]
 \draw [line width=3pt, line cap=round] (0.75,0.75) -- (0.25,0.75); 
 \draw [line width=3pt, line cap=round] (0.25,0.75) -- (0.25,0.25); 
 \draw [line width=3pt, line cap=round] (0.25,0.25) -- (0.75,0.25); 
 \draw [line width=3pt, line cap=round] (1,1.5) -- (0.75,1.25); 
 \draw [line width=3pt, line cap=round] (0.75,1.75) -- (0.25,1.75); 
 \draw [line width=3pt, line cap=round] (0.75,1.75) -- (0.875,1.625); 
 \draw [line width=3pt, line cap=round] (0.25,1.75) -- (0.25,1.25); 
 \draw [line width=3pt, line cap=round] (0.25,1.25) -- (0.75,1.25); 
 \draw [line width=3pt, line cap=round] (1.25,0.25) -- (0.75,0.25); 
 \draw [line width=3pt, line cap=round] (1,1.5) -- (0.75,1.25); 
 \draw [line width=3pt, line cap=round] (0.75,0.75) -- (1.25,0.75); 
 \draw [line width=3pt, line cap=round] (1,1.5) -- (1.25,1.75); 
 \draw [line width=3pt, line cap=round] (1.75,0.25) -- (1.75,0.75); 
 \draw [line width=3pt, line cap=round] (1.75,0.75) -- (1.25,0.75); 
 \draw [line width=3pt, line cap=round] (1.25,0.25) -- (1.75,0.25); 
 \draw [line width=3pt, line cap=round] (1.75,1.25) -- (1.75,1.75); 
 \draw [line width=3pt, line cap=round] (1.75,1.75) -- (1.25,1.75); 
 \draw [line width=3pt, line cap=round] (1,1.5) -- (1.25,1.75); 
 \draw [line width=3pt, line cap=round] (1.25,1.25) -- (1.75,1.25); 
 \draw [line width=3pt, line cap=round] (1.25,1.25) -- (1.125,1.375); 
\end{tikzpicture} 

\begin{tikzpicture}[framed,background rectangle/.style={ultra thick,draw=black}]
 \draw [line width=3pt, line cap=round] (0.25,0.75) -- (0.25,1.25); 
 \draw [line width=3pt, line cap=round] (0.25,0.75) -- (0.25,0.25); 
 \draw [line width=3pt, line cap=round] (0.25,0.25) -- (0.75,0.25); 
 \draw [line width=3pt, line cap=round] (1,1.5) -- (0.75,1.25); 
 \draw [line width=3pt, line cap=round] (0.75,1.75) -- (0.25,1.75); 
 \draw [line width=3pt, line cap=round] (0.75,1.75) -- (0.875,1.625); 
 \draw [line width=3pt, line cap=round] (0.25,1.75) -- (0.25,1.25); 
 \draw [line width=3pt, line cap=round] (1.25,0.25) -- (0.75,0.25); 
 \draw [line width=3pt, line cap=round] (1,1.5) -- (0.75,1.25); 
 \draw [line width=3pt, line cap=round] (0.75,1.25) -- (0.75,0.75); 
 \draw [line width=3pt, line cap=round] (0.75,0.75) -- (1.25,0.75); 
 \draw [line width=3pt, line cap=round] (1,1.5) -- (1.25,1.75); 
 \draw [line width=3pt, line cap=round] (1.75,0.25) -- (1.75,0.75); 
 \draw [line width=3pt, line cap=round] (1.75,0.75) -- (1.25,0.75); 
 \draw [line width=3pt, line cap=round] (1.25,0.25) -- (1.75,0.25); 
 \draw [line width=3pt, line cap=round] (1.75,1.25) -- (1.75,1.75); 
 \draw [line width=3pt, line cap=round] (1.75,1.75) -- (1.25,1.75); 
 \draw [line width=3pt, line cap=round] (1,1.5) -- (1.25,1.75); 
 \draw [line width=3pt, line cap=round] (1.25,1.25) -- (1.75,1.25); 
 \draw [line width=3pt, line cap=round] (1.25,1.25) -- (1.125,1.375); 
\end{tikzpicture} 

\begin{tikzpicture}[framed,background rectangle/.style={ultra thick,draw=black}]
 \draw [line width=3pt, line cap=round] (0.5,1) -- (0.25,1.25); 
 \draw [line width=3pt, line cap=round] (0.5,1) -- (0.75,0.75); 
 \draw [line width=3pt, line cap=round] (0.25,0.75) -- (0.25,0.25); 
 \draw [line width=3pt, line cap=round] (0.25,0.75) -- (0.375,0.875); 
 \draw [line width=3pt, line cap=round] (0.25,0.25) -- (0.75,0.25); 
 \draw [line width=3pt, line cap=round] (0.75,1.75) -- (0.25,1.75); 
 \draw [line width=3pt, line cap=round] (0.25,1.75) -- (0.25,1.25); 
 \draw [line width=3pt, line cap=round] (0.5,1) -- (0.25,1.25); 
 \draw [line width=3pt, line cap=round] (1.25,0.25) -- (0.75,0.25); 
 \draw [line width=3pt, line cap=round] (1.25,1.25) -- (0.75,1.25); 
 \draw [line width=3pt, line cap=round] (0.75,1.25) -- (0.625,1.125); 
 \draw [line width=3pt, line cap=round] (0.5,1) -- (0.75,0.75); 
 \draw [line width=3pt, line cap=round] (0.75,0.75) -- (1.25,0.75); 
 \draw [line width=3pt, line cap=round] (0.75,1.75) -- (1.25,1.75); 
 \draw [line width=3pt, line cap=round] (1.75,0.25) -- (1.75,0.75); 
 \draw [line width=3pt, line cap=round] (1.75,0.75) -- (1.25,0.75); 
 \draw [line width=3pt, line cap=round] (1.25,0.25) -- (1.75,0.25); 
 \draw [line width=3pt, line cap=round] (1.75,1.25) -- (1.75,1.75); 
 \draw [line width=3pt, line cap=round] (1.75,1.75) -- (1.25,1.75); 
 \draw [line width=3pt, line cap=round] (1.25,1.25) -- (1.75,1.25); 
\end{tikzpicture} 


\noindent
\begin{tikzpicture}[framed,background rectangle/.style={ultra thick,draw=black}]
 \draw [line width=3pt, line cap=round] (0.75,0.75) -- (0.25,0.75); 
 \draw [line width=3pt, line cap=round] (0.25,0.75) -- (0.25,0.25); 
 \draw [line width=3pt, line cap=round] (0.25,0.25) -- (0.75,0.25); 
 \draw [line width=3pt, line cap=round] (0.75,1.75) -- (0.25,1.75); 
 \draw [line width=3pt, line cap=round] (0.25,1.75) -- (0.25,1.25); 
 \draw [line width=3pt, line cap=round] (0.25,1.25) -- (0.75,1.25); 
 \draw [line width=3pt, line cap=round] (1.25,0.25) -- (0.75,0.25); 
 \draw [line width=3pt, line cap=round] (1.25,1.25) -- (0.75,1.25); 
 \draw [line width=3pt, line cap=round] (0.75,0.75) -- (1.25,0.75); 
 \draw [line width=3pt, line cap=round] (0.75,1.75) -- (1.25,1.75); 
 \draw [line width=3pt, line cap=round] (1.75,0.25) -- (1.75,0.75); 
 \draw [line width=3pt, line cap=round] (1.75,0.75) -- (1.25,0.75); 
 \draw [line width=3pt, line cap=round] (1.25,0.25) -- (1.75,0.25); 
 \draw [line width=3pt, line cap=round] (1.75,1.25) -- (1.75,1.75); 
 \draw [line width=3pt, line cap=round] (1.75,1.75) -- (1.25,1.75); 
 \draw [line width=3pt, line cap=round] (1.25,1.25) -- (1.75,1.25); 
\end{tikzpicture} 

\begin{tikzpicture}[framed,background rectangle/.style={ultra thick,draw=black}]
 \draw [line width=3pt, line cap=round] (0.25,0.75) -- (0.25,1.25); 
 \draw [line width=3pt, line cap=round] (0.25,0.75) -- (0.25,0.25); 
 \draw [line width=3pt, line cap=round] (0.25,0.25) -- (0.75,0.25); 
 \draw [line width=3pt, line cap=round] (0.75,1.75) -- (0.25,1.75); 
 \draw [line width=3pt, line cap=round] (0.25,1.75) -- (0.25,1.25); 
 \draw [line width=3pt, line cap=round] (1.25,0.25) -- (0.75,0.25); 
 \draw [line width=3pt, line cap=round] (1.25,1.25) -- (0.75,1.25); 
 \draw [line width=3pt, line cap=round] (0.75,1.25) -- (0.75,0.75); 
 \draw [line width=3pt, line cap=round] (0.75,0.75) -- (1.25,0.75); 
 \draw [line width=3pt, line cap=round] (0.75,1.75) -- (1.25,1.75); 
 \draw [line width=3pt, line cap=round] (1.75,0.25) -- (1.75,0.75); 
 \draw [line width=3pt, line cap=round] (1.75,0.75) -- (1.25,0.75); 
 \draw [line width=3pt, line cap=round] (1.25,0.25) -- (1.75,0.25); 
 \draw [line width=3pt, line cap=round] (1.75,1.25) -- (1.75,1.75); 
 \draw [line width=3pt, line cap=round] (1.75,1.75) -- (1.25,1.75); 
 \draw [line width=3pt, line cap=round] (1.25,1.25) -- (1.75,1.25); 
\end{tikzpicture} 

\begin{tikzpicture}[framed,background rectangle/.style={ultra thick,draw=black}]
 \draw [line width=3pt, line cap=round] (0.5,1) -- (0.25,1.25); 
 \draw [line width=3pt, line cap=round] (0.5,1) -- (0.75,0.75); 
 \draw [line width=3pt, line cap=round] (0.25,0.75) -- (0.25,0.25); 
 \draw [line width=3pt, line cap=round] (0.25,0.75) -- (0.375,0.875); 
 \draw [line width=3pt, line cap=round] (0.25,0.25) -- (0.75,0.25); 
 \draw [line width=3pt, line cap=round] (0.75,1.25) -- (0.75,1.75); 
 \draw [line width=3pt, line cap=round] (0.75,1.25) -- (0.625,1.125); 
 \draw [line width=3pt, line cap=round] (0.75,1.75) -- (0.25,1.75); 
 \draw [line width=3pt, line cap=round] (0.25,1.75) -- (0.25,1.25); 
 \draw [line width=3pt, line cap=round] (0.5,1) -- (0.25,1.25); 
 \draw [line width=3pt, line cap=round] (1.25,0.25) -- (0.75,0.25); 
 \draw [line width=3pt, line cap=round] (0.5,1) -- (0.75,0.75); 
 \draw [line width=3pt, line cap=round] (0.75,0.75) -- (1.25,0.75); 
 \draw [line width=3pt, line cap=round] (1.75,0.25) -- (1.75,0.75); 
 \draw [line width=3pt, line cap=round] (1.75,0.75) -- (1.25,0.75); 
 \draw [line width=3pt, line cap=round] (1.25,0.25) -- (1.75,0.25); 
 \draw [line width=3pt, line cap=round] (1.75,1.25) -- (1.75,1.75); 
 \draw [line width=3pt, line cap=round] (1.75,1.75) -- (1.25,1.75); 
 \draw [line width=3pt, line cap=round] (1.25,1.75) -- (1.25,1.25); 
 \draw [line width=3pt, line cap=round] (1.25,1.25) -- (1.75,1.25); 
\end{tikzpicture} 

\begin{tikzpicture}[framed,background rectangle/.style={ultra thick,draw=black}]
 \draw [line width=3pt, line cap=round] (0.75,0.75) -- (0.25,0.75); 
 \draw [line width=3pt, line cap=round] (0.25,0.75) -- (0.25,0.25); 
 \draw [line width=3pt, line cap=round] (0.25,0.25) -- (0.75,0.25); 
 \draw [line width=3pt, line cap=round] (0.75,1.25) -- (0.75,1.75); 
 \draw [line width=3pt, line cap=round] (0.75,1.75) -- (0.25,1.75); 
 \draw [line width=3pt, line cap=round] (0.25,1.75) -- (0.25,1.25); 
 \draw [line width=3pt, line cap=round] (0.25,1.25) -- (0.75,1.25); 
 \draw [line width=3pt, line cap=round] (1.25,0.25) -- (0.75,0.25); 
 \draw [line width=3pt, line cap=round] (0.75,0.75) -- (1.25,0.75); 
 \draw [line width=3pt, line cap=round] (1.75,0.25) -- (1.75,0.75); 
 \draw [line width=3pt, line cap=round] (1.75,0.75) -- (1.25,0.75); 
 \draw [line width=3pt, line cap=round] (1.25,0.25) -- (1.75,0.25); 
 \draw [line width=3pt, line cap=round] (1.75,1.25) -- (1.75,1.75); 
 \draw [line width=3pt, line cap=round] (1.75,1.75) -- (1.25,1.75); 
 \draw [line width=3pt, line cap=round] (1.25,1.75) -- (1.25,1.25); 
 \draw [line width=3pt, line cap=round] (1.25,1.25) -- (1.75,1.25); 
\end{tikzpicture} 

\begin{tikzpicture}[framed,background rectangle/.style={ultra thick,draw=black}]
 \draw [line width=3pt, line cap=round] (0.25,0.75) -- (0.25,1.25); 
 \draw [line width=3pt, line cap=round] (0.25,0.75) -- (0.25,0.25); 
 \draw [line width=3pt, line cap=round] (0.25,0.25) -- (0.75,0.25); 
 \draw [line width=3pt, line cap=round] (0.75,1.25) -- (0.75,1.75); 
 \draw [line width=3pt, line cap=round] (0.75,1.75) -- (0.25,1.75); 
 \draw [line width=3pt, line cap=round] (0.25,1.75) -- (0.25,1.25); 
 \draw [line width=3pt, line cap=round] (1.25,0.25) -- (0.75,0.25); 
 \draw [line width=3pt, line cap=round] (0.75,1.25) -- (0.75,0.75); 
 \draw [line width=3pt, line cap=round] (0.75,0.75) -- (1.25,0.75); 
 \draw [line width=3pt, line cap=round] (1.75,0.25) -- (1.75,0.75); 
 \draw [line width=3pt, line cap=round] (1.75,0.75) -- (1.25,0.75); 
 \draw [line width=3pt, line cap=round] (1.25,0.25) -- (1.75,0.25); 
 \draw [line width=3pt, line cap=round] (1.75,1.25) -- (1.75,1.75); 
 \draw [line width=3pt, line cap=round] (1.75,1.75) -- (1.25,1.75); 
 \draw [line width=3pt, line cap=round] (1.25,1.75) -- (1.25,1.25); 
 \draw [line width=3pt, line cap=round] (1.25,1.25) -- (1.75,1.25); 
\end{tikzpicture} 


\noindent
\begin{tikzpicture}[framed,background rectangle/.style={ultra thick,draw=black}]
 \draw [line width=3pt, line cap=round] (0.5,1) -- (0.25,1.25); 
 \draw [line width=3pt, line cap=round] (0.5,1) -- (0.75,0.75); 
 \draw [line width=3pt, line cap=round] (0.25,0.75) -- (0.25,0.25); 
 \draw [line width=3pt, line cap=round] (0.25,0.75) -- (0.375,0.875); 
 \draw [line width=3pt, line cap=round] (0.25,0.25) -- (0.75,0.25); 
 \draw [line width=3pt, line cap=round] (1,1.5) -- (0.75,1.25); 
 \draw [line width=3pt, line cap=round] (0.75,1.25) -- (0.625,1.125); 
 \draw [line width=3pt, line cap=round] (0.75,1.75) -- (0.25,1.75); 
 \draw [line width=3pt, line cap=round] (0.75,1.75) -- (0.875,1.625); 
 \draw [line width=3pt, line cap=round] (0.25,1.75) -- (0.25,1.25); 
 \draw [line width=3pt, line cap=round] (0.5,1) -- (0.25,1.25); 
 \draw [line width=3pt, line cap=round] (1.25,0.25) -- (0.75,0.25); 
 \draw [line width=3pt, line cap=round] (1.25,0.75) -- (1.25,1.25); 
 \draw [line width=3pt, line cap=round] (1.25,1.25) -- (1.125,1.375); 
 \draw [line width=3pt, line cap=round] (1,1.5) -- (0.75,1.25); 
 \draw [line width=3pt, line cap=round] (0.5,1) -- (0.75,0.75); 
 \draw [line width=3pt, line cap=round] (0.75,0.75) -- (1.25,0.75); 
 \draw [line width=3pt, line cap=round] (1,1.5) -- (1.25,1.75); 
 \draw [line width=3pt, line cap=round] (1.75,0.25) -- (1.75,0.75); 
 \draw [line width=3pt, line cap=round] (1.25,0.25) -- (1.75,0.25); 
 \draw [line width=3pt, line cap=round] (1.75,1.25) -- (1.75,1.75); 
 \draw [line width=3pt, line cap=round] (1.75,1.75) -- (1.25,1.75); 
 \draw [line width=3pt, line cap=round] (1,1.5) -- (1.25,1.75); 
 \draw [line width=3pt, line cap=round] (1.75,1.25) -- (1.75,0.75); 
\end{tikzpicture} 

\begin{tikzpicture}[framed,background rectangle/.style={ultra thick,draw=black}]
 \draw [line width=3pt, line cap=round] (0.75,0.75) -- (0.25,0.75); 
 \draw [line width=3pt, line cap=round] (0.25,0.75) -- (0.25,0.25); 
 \draw [line width=3pt, line cap=round] (0.25,0.25) -- (0.75,0.25); 
 \draw [line width=3pt, line cap=round] (1,1.5) -- (0.75,1.25); 
 \draw [line width=3pt, line cap=round] (0.75,1.75) -- (0.25,1.75); 
 \draw [line width=3pt, line cap=round] (0.75,1.75) -- (0.875,1.625); 
 \draw [line width=3pt, line cap=round] (0.25,1.75) -- (0.25,1.25); 
 \draw [line width=3pt, line cap=round] (0.25,1.25) -- (0.75,1.25); 
 \draw [line width=3pt, line cap=round] (1.25,0.25) -- (0.75,0.25); 
 \draw [line width=3pt, line cap=round] (1.25,0.75) -- (1.25,1.25); 
 \draw [line width=3pt, line cap=round] (1.25,1.25) -- (1.125,1.375); 
 \draw [line width=3pt, line cap=round] (1,1.5) -- (0.75,1.25); 
 \draw [line width=3pt, line cap=round] (0.75,0.75) -- (1.25,0.75); 
 \draw [line width=3pt, line cap=round] (1,1.5) -- (1.25,1.75); 
 \draw [line width=3pt, line cap=round] (1.75,0.25) -- (1.75,0.75); 
 \draw [line width=3pt, line cap=round] (1.25,0.25) -- (1.75,0.25); 
 \draw [line width=3pt, line cap=round] (1.75,1.25) -- (1.75,1.75); 
 \draw [line width=3pt, line cap=round] (1.75,1.75) -- (1.25,1.75); 
 \draw [line width=3pt, line cap=round] (1,1.5) -- (1.25,1.75); 
 \draw [line width=3pt, line cap=round] (1.75,1.25) -- (1.75,0.75); 
\end{tikzpicture} 

\begin{tikzpicture}[framed,background rectangle/.style={ultra thick,draw=black}]
 \draw [line width=3pt, line cap=round] (0.25,0.75) -- (0.25,1.25); 
 \draw [line width=3pt, line cap=round] (0.25,0.75) -- (0.25,0.25); 
 \draw [line width=3pt, line cap=round] (0.25,0.25) -- (0.75,0.25); 
 \draw [line width=3pt, line cap=round] (1,1.5) -- (0.75,1.25); 
 \draw [line width=3pt, line cap=round] (0.75,1.75) -- (0.25,1.75); 
 \draw [line width=3pt, line cap=round] (0.75,1.75) -- (0.875,1.625); 
 \draw [line width=3pt, line cap=round] (0.25,1.75) -- (0.25,1.25); 
 \draw [line width=3pt, line cap=round] (1.25,0.25) -- (0.75,0.25); 
 \draw [line width=3pt, line cap=round] (1.25,0.75) -- (1.25,1.25); 
 \draw [line width=3pt, line cap=round] (1.25,1.25) -- (1.125,1.375); 
 \draw [line width=3pt, line cap=round] (1,1.5) -- (0.75,1.25); 
 \draw [line width=3pt, line cap=round] (0.75,1.25) -- (0.75,0.75); 
 \draw [line width=3pt, line cap=round] (0.75,0.75) -- (1.25,0.75); 
 \draw [line width=3pt, line cap=round] (1,1.5) -- (1.25,1.75); 
 \draw [line width=3pt, line cap=round] (1.75,0.25) -- (1.75,0.75); 
 \draw [line width=3pt, line cap=round] (1.25,0.25) -- (1.75,0.25); 
 \draw [line width=3pt, line cap=round] (1.75,1.25) -- (1.75,1.75); 
 \draw [line width=3pt, line cap=round] (1.75,1.75) -- (1.25,1.75); 
 \draw [line width=3pt, line cap=round] (1,1.5) -- (1.25,1.75); 
 \draw [line width=3pt, line cap=round] (1.75,1.25) -- (1.75,0.75); 
\end{tikzpicture} 

\begin{tikzpicture}[framed,background rectangle/.style={ultra thick,draw=black}]
 \draw [line width=3pt, line cap=round] (0.5,1) -- (0.25,1.25); 
 \draw [line width=3pt, line cap=round] (0.5,1) -- (0.75,0.75); 
 \draw [line width=3pt, line cap=round] (0.25,0.75) -- (0.25,0.25); 
 \draw [line width=3pt, line cap=round] (0.25,0.75) -- (0.375,0.875); 
 \draw [line width=3pt, line cap=round] (0.25,0.25) -- (0.75,0.25); 
 \draw [line width=3pt, line cap=round] (0.75,1.75) -- (0.25,1.75); 
 \draw [line width=3pt, line cap=round] (0.25,1.75) -- (0.25,1.25); 
 \draw [line width=3pt, line cap=round] (0.5,1) -- (0.25,1.25); 
 \draw [line width=3pt, line cap=round] (1.25,0.25) -- (0.75,0.25); 
 \draw [line width=3pt, line cap=round] (1.25,0.75) -- (1.25,1.25); 
 \draw [line width=3pt, line cap=round] (1.25,1.25) -- (0.75,1.25); 
 \draw [line width=3pt, line cap=round] (0.75,1.25) -- (0.625,1.125); 
 \draw [line width=3pt, line cap=round] (0.5,1) -- (0.75,0.75); 
 \draw [line width=3pt, line cap=round] (0.75,0.75) -- (1.25,0.75); 
 \draw [line width=3pt, line cap=round] (0.75,1.75) -- (1.25,1.75); 
 \draw [line width=3pt, line cap=round] (1.75,0.25) -- (1.75,0.75); 
 \draw [line width=3pt, line cap=round] (1.25,0.25) -- (1.75,0.25); 
 \draw [line width=3pt, line cap=round] (1.75,1.25) -- (1.75,1.75); 
 \draw [line width=3pt, line cap=round] (1.75,1.75) -- (1.25,1.75); 
 \draw [line width=3pt, line cap=round] (1.75,1.25) -- (1.75,0.75); 
\end{tikzpicture} 

\begin{tikzpicture}[framed,background rectangle/.style={ultra thick,draw=black}]
 \draw [line width=3pt, line cap=round] (0.75,0.75) -- (0.25,0.75); 
 \draw [line width=3pt, line cap=round] (0.25,0.75) -- (0.25,0.25); 
 \draw [line width=3pt, line cap=round] (0.25,0.25) -- (0.75,0.25); 
 \draw [line width=3pt, line cap=round] (0.75,1.75) -- (0.25,1.75); 
 \draw [line width=3pt, line cap=round] (0.25,1.75) -- (0.25,1.25); 
 \draw [line width=3pt, line cap=round] (0.25,1.25) -- (0.75,1.25); 
 \draw [line width=3pt, line cap=round] (1.25,0.25) -- (0.75,0.25); 
 \draw [line width=3pt, line cap=round] (1.25,0.75) -- (1.25,1.25); 
 \draw [line width=3pt, line cap=round] (1.25,1.25) -- (0.75,1.25); 
 \draw [line width=3pt, line cap=round] (0.75,0.75) -- (1.25,0.75); 
 \draw [line width=3pt, line cap=round] (0.75,1.75) -- (1.25,1.75); 
 \draw [line width=3pt, line cap=round] (1.75,0.25) -- (1.75,0.75); 
 \draw [line width=3pt, line cap=round] (1.25,0.25) -- (1.75,0.25); 
 \draw [line width=3pt, line cap=round] (1.75,1.25) -- (1.75,1.75); 
 \draw [line width=3pt, line cap=round] (1.75,1.75) -- (1.25,1.75); 
 \draw [line width=3pt, line cap=round] (1.75,1.25) -- (1.75,0.75); 
\end{tikzpicture} 


\noindent
\begin{tikzpicture}[framed,background rectangle/.style={ultra thick,draw=black}]
 \draw [line width=3pt, line cap=round] (0.25,0.75) -- (0.25,1.25); 
 \draw [line width=3pt, line cap=round] (0.25,0.75) -- (0.25,0.25); 
 \draw [line width=3pt, line cap=round] (0.25,0.25) -- (0.75,0.25); 
 \draw [line width=3pt, line cap=round] (0.75,1.75) -- (0.25,1.75); 
 \draw [line width=3pt, line cap=round] (0.25,1.75) -- (0.25,1.25); 
 \draw [line width=3pt, line cap=round] (1.25,0.25) -- (0.75,0.25); 
 \draw [line width=3pt, line cap=round] (1.25,0.75) -- (1.25,1.25); 
 \draw [line width=3pt, line cap=round] (1.25,1.25) -- (0.75,1.25); 
 \draw [line width=3pt, line cap=round] (0.75,1.25) -- (0.75,0.75); 
 \draw [line width=3pt, line cap=round] (0.75,0.75) -- (1.25,0.75); 
 \draw [line width=3pt, line cap=round] (0.75,1.75) -- (1.25,1.75); 
 \draw [line width=3pt, line cap=round] (1.75,0.25) -- (1.75,0.75); 
 \draw [line width=3pt, line cap=round] (1.25,0.25) -- (1.75,0.25); 
 \draw [line width=3pt, line cap=round] (1.75,1.25) -- (1.75,1.75); 
 \draw [line width=3pt, line cap=round] (1.75,1.75) -- (1.25,1.75); 
 \draw [line width=3pt, line cap=round] (1.75,1.25) -- (1.75,0.75); 
\end{tikzpicture} 

\begin{tikzpicture}[framed,background rectangle/.style={ultra thick,draw=black}]
 \draw [line width=3pt, line cap=round] (0.5,1) -- (0.25,1.25); 
 \draw [line width=3pt, line cap=round] (0.5,1) -- (0.75,0.75); 
 \draw [line width=3pt, line cap=round] (0.25,0.75) -- (0.25,0.25); 
 \draw [line width=3pt, line cap=round] (0.25,0.75) -- (0.375,0.875); 
 \draw [line width=3pt, line cap=round] (0.25,0.25) -- (0.75,0.25); 
 \draw [line width=3pt, line cap=round] (0.75,1.25) -- (0.75,1.75); 
 \draw [line width=3pt, line cap=round] (0.75,1.25) -- (0.625,1.125); 
 \draw [line width=3pt, line cap=round] (0.75,1.75) -- (0.25,1.75); 
 \draw [line width=3pt, line cap=round] (0.25,1.75) -- (0.25,1.25); 
 \draw [line width=3pt, line cap=round] (0.5,1) -- (0.25,1.25); 
 \draw [line width=3pt, line cap=round] (1.25,0.25) -- (0.75,0.25); 
 \draw [line width=3pt, line cap=round] (1.25,0.75) -- (1.25,1.25); 
 \draw [line width=3pt, line cap=round] (0.5,1) -- (0.75,0.75); 
 \draw [line width=3pt, line cap=round] (0.75,0.75) -- (1.25,0.75); 
 \draw [line width=3pt, line cap=round] (1.75,0.25) -- (1.75,0.75); 
 \draw [line width=3pt, line cap=round] (1.25,0.25) -- (1.75,0.25); 
 \draw [line width=3pt, line cap=round] (1.75,1.25) -- (1.75,1.75); 
 \draw [line width=3pt, line cap=round] (1.75,1.75) -- (1.25,1.75); 
 \draw [line width=3pt, line cap=round] (1.25,1.75) -- (1.25,1.25); 
 \draw [line width=3pt, line cap=round] (1.75,1.25) -- (1.75,0.75); 
\end{tikzpicture} 

\begin{tikzpicture}[framed,background rectangle/.style={ultra thick,draw=black}]
 \draw [line width=3pt, line cap=round] (0.75,0.75) -- (0.25,0.75); 
 \draw [line width=3pt, line cap=round] (0.25,0.75) -- (0.25,0.25); 
 \draw [line width=3pt, line cap=round] (0.25,0.25) -- (0.75,0.25); 
 \draw [line width=3pt, line cap=round] (0.75,1.25) -- (0.75,1.75); 
 \draw [line width=3pt, line cap=round] (0.75,1.75) -- (0.25,1.75); 
 \draw [line width=3pt, line cap=round] (0.25,1.75) -- (0.25,1.25); 
 \draw [line width=3pt, line cap=round] (0.25,1.25) -- (0.75,1.25); 
 \draw [line width=3pt, line cap=round] (1.25,0.25) -- (0.75,0.25); 
 \draw [line width=3pt, line cap=round] (1.25,0.75) -- (1.25,1.25); 
 \draw [line width=3pt, line cap=round] (0.75,0.75) -- (1.25,0.75); 
 \draw [line width=3pt, line cap=round] (1.75,0.25) -- (1.75,0.75); 
 \draw [line width=3pt, line cap=round] (1.25,0.25) -- (1.75,0.25); 
 \draw [line width=3pt, line cap=round] (1.75,1.25) -- (1.75,1.75); 
 \draw [line width=3pt, line cap=round] (1.75,1.75) -- (1.25,1.75); 
 \draw [line width=3pt, line cap=round] (1.25,1.75) -- (1.25,1.25); 
 \draw [line width=3pt, line cap=round] (1.75,1.25) -- (1.75,0.75); 
\end{tikzpicture} 

\begin{tikzpicture}[framed,background rectangle/.style={ultra thick,draw=black}]
 \draw [line width=3pt, line cap=round] (0.25,0.75) -- (0.25,1.25); 
 \draw [line width=3pt, line cap=round] (0.25,0.75) -- (0.25,0.25); 
 \draw [line width=3pt, line cap=round] (0.25,0.25) -- (0.75,0.25); 
 \draw [line width=3pt, line cap=round] (0.75,1.25) -- (0.75,1.75); 
 \draw [line width=3pt, line cap=round] (0.75,1.75) -- (0.25,1.75); 
 \draw [line width=3pt, line cap=round] (0.25,1.75) -- (0.25,1.25); 
 \draw [line width=3pt, line cap=round] (1.25,0.25) -- (0.75,0.25); 
 \draw [line width=3pt, line cap=round] (1.25,0.75) -- (1.25,1.25); 
 \draw [line width=3pt, line cap=round] (0.75,1.25) -- (0.75,0.75); 
 \draw [line width=3pt, line cap=round] (0.75,0.75) -- (1.25,0.75); 
 \draw [line width=3pt, line cap=round] (1.75,0.25) -- (1.75,0.75); 
 \draw [line width=3pt, line cap=round] (1.25,0.25) -- (1.75,0.25); 
 \draw [line width=3pt, line cap=round] (1.75,1.25) -- (1.75,1.75); 
 \draw [line width=3pt, line cap=round] (1.75,1.75) -- (1.25,1.75); 
 \draw [line width=3pt, line cap=round] (1.25,1.75) -- (1.25,1.25); 
 \draw [line width=3pt, line cap=round] (1.75,1.25) -- (1.75,0.75); 
\end{tikzpicture} 

\begin{tikzpicture}[framed,background rectangle/.style={ultra thick,draw=black}]
 \draw [line width=3pt, line cap=round] (0.5,1) -- (0.25,1.25); 
 \draw [line width=3pt, line cap=round] (0.75,0.25) -- (0.75,0.75); 
 \draw [line width=3pt, line cap=round] (0.5,1) -- (0.75,0.75); 
 \draw [line width=3pt, line cap=round] (0.25,0.75) -- (0.25,0.25); 
 \draw [line width=3pt, line cap=round] (0.25,0.75) -- (0.375,0.875); 
 \draw [line width=3pt, line cap=round] (0.25,0.25) -- (0.75,0.25); 
 \draw [line width=3pt, line cap=round] (1,1.5) -- (0.75,1.25); 
 \draw [line width=3pt, line cap=round] (0.75,1.25) -- (0.625,1.125); 
 \draw [line width=3pt, line cap=round] (0.75,1.75) -- (0.25,1.75); 
 \draw [line width=3pt, line cap=round] (0.75,1.75) -- (0.875,1.625); 
 \draw [line width=3pt, line cap=round] (0.25,1.75) -- (0.25,1.25); 
 \draw [line width=3pt, line cap=round] (0.5,1) -- (0.25,1.25); 
 \draw [line width=3pt, line cap=round] (1.5,1) -- (1.25,1.25); 
 \draw [line width=3pt, line cap=round] (1,1.5) -- (0.75,1.25); 
 \draw [line width=3pt, line cap=round] (0.5,1) -- (0.75,0.75); 
 \draw [line width=3pt, line cap=round] (1,1.5) -- (1.25,1.75); 
 \draw [line width=3pt, line cap=round] (1.75,0.25) -- (1.75,0.75); 
 \draw [line width=3pt, line cap=round] (1.5,1) -- (1.75,0.75); 
 \draw [line width=3pt, line cap=round] (1.25,0.75) -- (1.25,0.25); 
 \draw [line width=3pt, line cap=round] (1.25,0.75) -- (1.375,0.875); 
 \draw [line width=3pt, line cap=round] (1.25,0.25) -- (1.75,0.25); 
 \draw [line width=3pt, line cap=round] (1.75,1.25) -- (1.75,1.75); 
 \draw [line width=3pt, line cap=round] (1.75,1.25) -- (1.625,1.125); 
 \draw [line width=3pt, line cap=round] (1.75,1.75) -- (1.25,1.75); 
 \draw [line width=3pt, line cap=round] (1,1.5) -- (1.25,1.75); 
 \draw [line width=3pt, line cap=round] (1.5,1) -- (1.25,1.25); 
 \draw [line width=3pt, line cap=round] (1.25,1.25) -- (1.125,1.375); 
 \draw [line width=3pt, line cap=round] (1.5,1) -- (1.75,0.75); 
\end{tikzpicture} 


\noindent
\begin{tikzpicture}[framed,background rectangle/.style={ultra thick,draw=black}]
 \draw [line width=3pt, line cap=round] (0.75,0.25) -- (0.75,0.75); 
 \draw [line width=3pt, line cap=round] (0.75,0.75) -- (0.25,0.75); 
 \draw [line width=3pt, line cap=round] (0.25,0.75) -- (0.25,0.25); 
 \draw [line width=3pt, line cap=round] (0.25,0.25) -- (0.75,0.25); 
 \draw [line width=3pt, line cap=round] (1,1.5) -- (0.75,1.25); 
 \draw [line width=3pt, line cap=round] (0.75,1.75) -- (0.25,1.75); 
 \draw [line width=3pt, line cap=round] (0.75,1.75) -- (0.875,1.625); 
 \draw [line width=3pt, line cap=round] (0.25,1.75) -- (0.25,1.25); 
 \draw [line width=3pt, line cap=round] (0.25,1.25) -- (0.75,1.25); 
 \draw [line width=3pt, line cap=round] (1.5,1) -- (1.25,1.25); 
 \draw [line width=3pt, line cap=round] (1,1.5) -- (0.75,1.25); 
 \draw [line width=3pt, line cap=round] (1,1.5) -- (1.25,1.75); 
 \draw [line width=3pt, line cap=round] (1.75,0.25) -- (1.75,0.75); 
 \draw [line width=3pt, line cap=round] (1.5,1) -- (1.75,0.75); 
 \draw [line width=3pt, line cap=round] (1.25,0.75) -- (1.25,0.25); 
 \draw [line width=3pt, line cap=round] (1.25,0.75) -- (1.375,0.875); 
 \draw [line width=3pt, line cap=round] (1.25,0.25) -- (1.75,0.25); 
 \draw [line width=3pt, line cap=round] (1.75,1.25) -- (1.75,1.75); 
 \draw [line width=3pt, line cap=round] (1.75,1.25) -- (1.625,1.125); 
 \draw [line width=3pt, line cap=round] (1.75,1.75) -- (1.25,1.75); 
 \draw [line width=3pt, line cap=round] (1,1.5) -- (1.25,1.75); 
 \draw [line width=3pt, line cap=round] (1.5,1) -- (1.25,1.25); 
 \draw [line width=3pt, line cap=round] (1.25,1.25) -- (1.125,1.375); 
 \draw [line width=3pt, line cap=round] (1.5,1) -- (1.75,0.75); 
\end{tikzpicture} 

\begin{tikzpicture}[framed,background rectangle/.style={ultra thick,draw=black}]
 \draw [line width=3pt, line cap=round] (0.25,0.75) -- (0.25,1.25); 
 \draw [line width=3pt, line cap=round] (0.75,0.25) -- (0.75,0.75); 
 \draw [line width=3pt, line cap=round] (0.25,0.75) -- (0.25,0.25); 
 \draw [line width=3pt, line cap=round] (0.25,0.25) -- (0.75,0.25); 
 \draw [line width=3pt, line cap=round] (1,1.5) -- (0.75,1.25); 
 \draw [line width=3pt, line cap=round] (0.75,1.75) -- (0.25,1.75); 
 \draw [line width=3pt, line cap=round] (0.75,1.75) -- (0.875,1.625); 
 \draw [line width=3pt, line cap=round] (0.25,1.75) -- (0.25,1.25); 
 \draw [line width=3pt, line cap=round] (1.5,1) -- (1.25,1.25); 
 \draw [line width=3pt, line cap=round] (1,1.5) -- (0.75,1.25); 
 \draw [line width=3pt, line cap=round] (0.75,1.25) -- (0.75,0.75); 
 \draw [line width=3pt, line cap=round] (1,1.5) -- (1.25,1.75); 
 \draw [line width=3pt, line cap=round] (1.75,0.25) -- (1.75,0.75); 
 \draw [line width=3pt, line cap=round] (1.5,1) -- (1.75,0.75); 
 \draw [line width=3pt, line cap=round] (1.25,0.75) -- (1.25,0.25); 
 \draw [line width=3pt, line cap=round] (1.25,0.75) -- (1.375,0.875); 
 \draw [line width=3pt, line cap=round] (1.25,0.25) -- (1.75,0.25); 
 \draw [line width=3pt, line cap=round] (1.75,1.25) -- (1.75,1.75); 
 \draw [line width=3pt, line cap=round] (1.75,1.25) -- (1.625,1.125); 
 \draw [line width=3pt, line cap=round] (1.75,1.75) -- (1.25,1.75); 
 \draw [line width=3pt, line cap=round] (1,1.5) -- (1.25,1.75); 
 \draw [line width=3pt, line cap=round] (1.5,1) -- (1.25,1.25); 
 \draw [line width=3pt, line cap=round] (1.25,1.25) -- (1.125,1.375); 
 \draw [line width=3pt, line cap=round] (1.5,1) -- (1.75,0.75); 
\end{tikzpicture} 

\begin{tikzpicture}[framed,background rectangle/.style={ultra thick,draw=black}]
 \draw [line width=3pt, line cap=round] (0.5,1) -- (0.25,1.25); 
 \draw [line width=3pt, line cap=round] (0.75,0.25) -- (0.75,0.75); 
 \draw [line width=3pt, line cap=round] (0.5,1) -- (0.75,0.75); 
 \draw [line width=3pt, line cap=round] (0.25,0.75) -- (0.25,0.25); 
 \draw [line width=3pt, line cap=round] (0.25,0.75) -- (0.375,0.875); 
 \draw [line width=3pt, line cap=round] (0.25,0.25) -- (0.75,0.25); 
 \draw [line width=3pt, line cap=round] (0.75,1.75) -- (0.25,1.75); 
 \draw [line width=3pt, line cap=round] (0.25,1.75) -- (0.25,1.25); 
 \draw [line width=3pt, line cap=round] (0.5,1) -- (0.25,1.25); 
 \draw [line width=3pt, line cap=round] (1.5,1) -- (1.25,1.25); 
 \draw [line width=3pt, line cap=round] (1.25,1.25) -- (0.75,1.25); 
 \draw [line width=3pt, line cap=round] (0.75,1.25) -- (0.625,1.125); 
 \draw [line width=3pt, line cap=round] (0.5,1) -- (0.75,0.75); 
 \draw [line width=3pt, line cap=round] (0.75,1.75) -- (1.25,1.75); 
 \draw [line width=3pt, line cap=round] (1.75,0.25) -- (1.75,0.75); 
 \draw [line width=3pt, line cap=round] (1.5,1) -- (1.75,0.75); 
 \draw [line width=3pt, line cap=round] (1.25,0.75) -- (1.25,0.25); 
 \draw [line width=3pt, line cap=round] (1.25,0.75) -- (1.375,0.875); 
 \draw [line width=3pt, line cap=round] (1.25,0.25) -- (1.75,0.25); 
 \draw [line width=3pt, line cap=round] (1.75,1.25) -- (1.75,1.75); 
 \draw [line width=3pt, line cap=round] (1.75,1.25) -- (1.625,1.125); 
 \draw [line width=3pt, line cap=round] (1.75,1.75) -- (1.25,1.75); 
 \draw [line width=3pt, line cap=round] (1.5,1) -- (1.25,1.25); 
 \draw [line width=3pt, line cap=round] (1.5,1) -- (1.75,0.75); 
\end{tikzpicture} 

\begin{tikzpicture}[framed,background rectangle/.style={ultra thick,draw=black}]
 \draw [line width=3pt, line cap=round] (0.75,0.25) -- (0.75,0.75); 
 \draw [line width=3pt, line cap=round] (0.75,0.75) -- (0.25,0.75); 
 \draw [line width=3pt, line cap=round] (0.25,0.75) -- (0.25,0.25); 
 \draw [line width=3pt, line cap=round] (0.25,0.25) -- (0.75,0.25); 
 \draw [line width=3pt, line cap=round] (0.75,1.75) -- (0.25,1.75); 
 \draw [line width=3pt, line cap=round] (0.25,1.75) -- (0.25,1.25); 
 \draw [line width=3pt, line cap=round] (0.25,1.25) -- (0.75,1.25); 
 \draw [line width=3pt, line cap=round] (1.5,1) -- (1.25,1.25); 
 \draw [line width=3pt, line cap=round] (1.25,1.25) -- (0.75,1.25); 
 \draw [line width=3pt, line cap=round] (0.75,1.75) -- (1.25,1.75); 
 \draw [line width=3pt, line cap=round] (1.75,0.25) -- (1.75,0.75); 
 \draw [line width=3pt, line cap=round] (1.5,1) -- (1.75,0.75); 
 \draw [line width=3pt, line cap=round] (1.25,0.75) -- (1.25,0.25); 
 \draw [line width=3pt, line cap=round] (1.25,0.75) -- (1.375,0.875); 
 \draw [line width=3pt, line cap=round] (1.25,0.25) -- (1.75,0.25); 
 \draw [line width=3pt, line cap=round] (1.75,1.25) -- (1.75,1.75); 
 \draw [line width=3pt, line cap=round] (1.75,1.25) -- (1.625,1.125); 
 \draw [line width=3pt, line cap=round] (1.75,1.75) -- (1.25,1.75); 
 \draw [line width=3pt, line cap=round] (1.5,1) -- (1.25,1.25); 
 \draw [line width=3pt, line cap=round] (1.5,1) -- (1.75,0.75); 
\end{tikzpicture} 

\begin{tikzpicture}[framed,background rectangle/.style={ultra thick,draw=black}]
 \draw [line width=3pt, line cap=round] (0.25,0.75) -- (0.25,1.25); 
 \draw [line width=3pt, line cap=round] (0.75,0.25) -- (0.75,0.75); 
 \draw [line width=3pt, line cap=round] (0.25,0.75) -- (0.25,0.25); 
 \draw [line width=3pt, line cap=round] (0.25,0.25) -- (0.75,0.25); 
 \draw [line width=3pt, line cap=round] (0.75,1.75) -- (0.25,1.75); 
 \draw [line width=3pt, line cap=round] (0.25,1.75) -- (0.25,1.25); 
 \draw [line width=3pt, line cap=round] (1.5,1) -- (1.25,1.25); 
 \draw [line width=3pt, line cap=round] (1.25,1.25) -- (0.75,1.25); 
 \draw [line width=3pt, line cap=round] (0.75,1.25) -- (0.75,0.75); 
 \draw [line width=3pt, line cap=round] (0.75,1.75) -- (1.25,1.75); 
 \draw [line width=3pt, line cap=round] (1.75,0.25) -- (1.75,0.75); 
 \draw [line width=3pt, line cap=round] (1.5,1) -- (1.75,0.75); 
 \draw [line width=3pt, line cap=round] (1.25,0.75) -- (1.25,0.25); 
 \draw [line width=3pt, line cap=round] (1.25,0.75) -- (1.375,0.875); 
 \draw [line width=3pt, line cap=round] (1.25,0.25) -- (1.75,0.25); 
 \draw [line width=3pt, line cap=round] (1.75,1.25) -- (1.75,1.75); 
 \draw [line width=3pt, line cap=round] (1.75,1.25) -- (1.625,1.125); 
 \draw [line width=3pt, line cap=round] (1.75,1.75) -- (1.25,1.75); 
 \draw [line width=3pt, line cap=round] (1.5,1) -- (1.25,1.25); 
 \draw [line width=3pt, line cap=round] (1.5,1) -- (1.75,0.75); 
\end{tikzpicture} 


\noindent
\begin{tikzpicture}[framed,background rectangle/.style={ultra thick,draw=black}]
 \draw [line width=3pt, line cap=round] (0.5,1) -- (0.25,1.25); 
 \draw [line width=3pt, line cap=round] (0.75,0.25) -- (0.75,0.75); 
 \draw [line width=3pt, line cap=round] (0.5,1) -- (0.75,0.75); 
 \draw [line width=3pt, line cap=round] (0.25,0.75) -- (0.25,0.25); 
 \draw [line width=3pt, line cap=round] (0.25,0.75) -- (0.375,0.875); 
 \draw [line width=3pt, line cap=round] (0.25,0.25) -- (0.75,0.25); 
 \draw [line width=3pt, line cap=round] (0.75,1.25) -- (0.75,1.75); 
 \draw [line width=3pt, line cap=round] (0.75,1.25) -- (0.625,1.125); 
 \draw [line width=3pt, line cap=round] (0.75,1.75) -- (0.25,1.75); 
 \draw [line width=3pt, line cap=round] (0.25,1.75) -- (0.25,1.25); 
 \draw [line width=3pt, line cap=round] (0.5,1) -- (0.25,1.25); 
 \draw [line width=3pt, line cap=round] (1.5,1) -- (1.25,1.25); 
 \draw [line width=3pt, line cap=round] (0.5,1) -- (0.75,0.75); 
 \draw [line width=3pt, line cap=round] (1.75,0.25) -- (1.75,0.75); 
 \draw [line width=3pt, line cap=round] (1.5,1) -- (1.75,0.75); 
 \draw [line width=3pt, line cap=round] (1.25,0.75) -- (1.25,0.25); 
 \draw [line width=3pt, line cap=round] (1.25,0.75) -- (1.375,0.875); 
 \draw [line width=3pt, line cap=round] (1.25,0.25) -- (1.75,0.25); 
 \draw [line width=3pt, line cap=round] (1.75,1.25) -- (1.75,1.75); 
 \draw [line width=3pt, line cap=round] (1.75,1.25) -- (1.625,1.125); 
 \draw [line width=3pt, line cap=round] (1.75,1.75) -- (1.25,1.75); 
 \draw [line width=3pt, line cap=round] (1.25,1.75) -- (1.25,1.25); 
 \draw [line width=3pt, line cap=round] (1.5,1) -- (1.25,1.25); 
 \draw [line width=3pt, line cap=round] (1.5,1) -- (1.75,0.75); 
\end{tikzpicture} 

\begin{tikzpicture}[framed,background rectangle/.style={ultra thick,draw=black}]
 \draw [line width=3pt, line cap=round] (0.75,0.25) -- (0.75,0.75); 
 \draw [line width=3pt, line cap=round] (0.75,0.75) -- (0.25,0.75); 
 \draw [line width=3pt, line cap=round] (0.25,0.75) -- (0.25,0.25); 
 \draw [line width=3pt, line cap=round] (0.25,0.25) -- (0.75,0.25); 
 \draw [line width=3pt, line cap=round] (0.75,1.25) -- (0.75,1.75); 
 \draw [line width=3pt, line cap=round] (0.75,1.75) -- (0.25,1.75); 
 \draw [line width=3pt, line cap=round] (0.25,1.75) -- (0.25,1.25); 
 \draw [line width=3pt, line cap=round] (0.25,1.25) -- (0.75,1.25); 
 \draw [line width=3pt, line cap=round] (1.5,1) -- (1.25,1.25); 
 \draw [line width=3pt, line cap=round] (1.75,0.25) -- (1.75,0.75); 
 \draw [line width=3pt, line cap=round] (1.5,1) -- (1.75,0.75); 
 \draw [line width=3pt, line cap=round] (1.25,0.75) -- (1.25,0.25); 
 \draw [line width=3pt, line cap=round] (1.25,0.75) -- (1.375,0.875); 
 \draw [line width=3pt, line cap=round] (1.25,0.25) -- (1.75,0.25); 
 \draw [line width=3pt, line cap=round] (1.75,1.25) -- (1.75,1.75); 
 \draw [line width=3pt, line cap=round] (1.75,1.25) -- (1.625,1.125); 
 \draw [line width=3pt, line cap=round] (1.75,1.75) -- (1.25,1.75); 
 \draw [line width=3pt, line cap=round] (1.25,1.75) -- (1.25,1.25); 
 \draw [line width=3pt, line cap=round] (1.5,1) -- (1.25,1.25); 
 \draw [line width=3pt, line cap=round] (1.5,1) -- (1.75,0.75); 
\end{tikzpicture} 

\begin{tikzpicture}[framed,background rectangle/.style={ultra thick,draw=black}]
 \draw [line width=3pt, line cap=round] (0.25,0.75) -- (0.25,1.25); 
 \draw [line width=3pt, line cap=round] (0.75,0.25) -- (0.75,0.75); 
 \draw [line width=3pt, line cap=round] (0.25,0.75) -- (0.25,0.25); 
 \draw [line width=3pt, line cap=round] (0.25,0.25) -- (0.75,0.25); 
 \draw [line width=3pt, line cap=round] (0.75,1.25) -- (0.75,1.75); 
 \draw [line width=3pt, line cap=round] (0.75,1.75) -- (0.25,1.75); 
 \draw [line width=3pt, line cap=round] (0.25,1.75) -- (0.25,1.25); 
 \draw [line width=3pt, line cap=round] (1.5,1) -- (1.25,1.25); 
 \draw [line width=3pt, line cap=round] (0.75,1.25) -- (0.75,0.75); 
 \draw [line width=3pt, line cap=round] (1.75,0.25) -- (1.75,0.75); 
 \draw [line width=3pt, line cap=round] (1.5,1) -- (1.75,0.75); 
 \draw [line width=3pt, line cap=round] (1.25,0.75) -- (1.25,0.25); 
 \draw [line width=3pt, line cap=round] (1.25,0.75) -- (1.375,0.875); 
 \draw [line width=3pt, line cap=round] (1.25,0.25) -- (1.75,0.25); 
 \draw [line width=3pt, line cap=round] (1.75,1.25) -- (1.75,1.75); 
 \draw [line width=3pt, line cap=round] (1.75,1.25) -- (1.625,1.125); 
 \draw [line width=3pt, line cap=round] (1.75,1.75) -- (1.25,1.75); 
 \draw [line width=3pt, line cap=round] (1.25,1.75) -- (1.25,1.25); 
 \draw [line width=3pt, line cap=round] (1.5,1) -- (1.25,1.25); 
 \draw [line width=3pt, line cap=round] (1.5,1) -- (1.75,0.75); 
\end{tikzpicture} 

\begin{tikzpicture}[framed,background rectangle/.style={ultra thick,draw=black}]
 \draw [line width=3pt, line cap=round] (0.5,1) -- (0.25,1.25); 
 \draw [line width=3pt, line cap=round] (0.75,0.25) -- (0.75,0.75); 
 \draw [line width=3pt, line cap=round] (0.5,1) -- (0.75,0.75); 
 \draw [line width=3pt, line cap=round] (0.25,0.75) -- (0.25,0.25); 
 \draw [line width=3pt, line cap=round] (0.25,0.75) -- (0.375,0.875); 
 \draw [line width=3pt, line cap=round] (0.25,0.25) -- (0.75,0.25); 
 \draw [line width=3pt, line cap=round] (1,1.5) -- (0.75,1.25); 
 \draw [line width=3pt, line cap=round] (0.75,1.25) -- (0.625,1.125); 
 \draw [line width=3pt, line cap=round] (0.75,1.75) -- (0.25,1.75); 
 \draw [line width=3pt, line cap=round] (0.75,1.75) -- (0.875,1.625); 
 \draw [line width=3pt, line cap=round] (0.25,1.75) -- (0.25,1.25); 
 \draw [line width=3pt, line cap=round] (0.5,1) -- (0.25,1.25); 
 \draw [line width=3pt, line cap=round] (1,1.5) -- (0.75,1.25); 
 \draw [line width=3pt, line cap=round] (0.5,1) -- (0.75,0.75); 
 \draw [line width=3pt, line cap=round] (1,1.5) -- (1.25,1.75); 
 \draw [line width=3pt, line cap=round] (1.75,0.25) -- (1.75,0.75); 
 \draw [line width=3pt, line cap=round] (1.75,0.75) -- (1.25,0.75); 
 \draw [line width=3pt, line cap=round] (1.25,0.75) -- (1.25,0.25); 
 \draw [line width=3pt, line cap=round] (1.25,0.25) -- (1.75,0.25); 
 \draw [line width=3pt, line cap=round] (1.75,1.25) -- (1.75,1.75); 
 \draw [line width=3pt, line cap=round] (1.75,1.75) -- (1.25,1.75); 
 \draw [line width=3pt, line cap=round] (1,1.5) -- (1.25,1.75); 
 \draw [line width=3pt, line cap=round] (1.25,1.25) -- (1.75,1.25); 
 \draw [line width=3pt, line cap=round] (1.25,1.25) -- (1.125,1.375); 
\end{tikzpicture} 

\begin{tikzpicture}[framed,background rectangle/.style={ultra thick,draw=black}]
 \draw [line width=3pt, line cap=round] (0.75,0.25) -- (0.75,0.75); 
 \draw [line width=3pt, line cap=round] (0.75,0.75) -- (0.25,0.75); 
 \draw [line width=3pt, line cap=round] (0.25,0.75) -- (0.25,0.25); 
 \draw [line width=3pt, line cap=round] (0.25,0.25) -- (0.75,0.25); 
 \draw [line width=3pt, line cap=round] (1,1.5) -- (0.75,1.25); 
 \draw [line width=3pt, line cap=round] (0.75,1.75) -- (0.25,1.75); 
 \draw [line width=3pt, line cap=round] (0.75,1.75) -- (0.875,1.625); 
 \draw [line width=3pt, line cap=round] (0.25,1.75) -- (0.25,1.25); 
 \draw [line width=3pt, line cap=round] (0.25,1.25) -- (0.75,1.25); 
 \draw [line width=3pt, line cap=round] (1,1.5) -- (0.75,1.25); 
 \draw [line width=3pt, line cap=round] (1,1.5) -- (1.25,1.75); 
 \draw [line width=3pt, line cap=round] (1.75,0.25) -- (1.75,0.75); 
 \draw [line width=3pt, line cap=round] (1.75,0.75) -- (1.25,0.75); 
 \draw [line width=3pt, line cap=round] (1.25,0.75) -- (1.25,0.25); 
 \draw [line width=3pt, line cap=round] (1.25,0.25) -- (1.75,0.25); 
 \draw [line width=3pt, line cap=round] (1.75,1.25) -- (1.75,1.75); 
 \draw [line width=3pt, line cap=round] (1.75,1.75) -- (1.25,1.75); 
 \draw [line width=3pt, line cap=round] (1,1.5) -- (1.25,1.75); 
 \draw [line width=3pt, line cap=round] (1.25,1.25) -- (1.75,1.25); 
 \draw [line width=3pt, line cap=round] (1.25,1.25) -- (1.125,1.375); 
\end{tikzpicture} 


\noindent
\begin{tikzpicture}[framed,background rectangle/.style={ultra thick,draw=black}]
 \draw [line width=3pt, line cap=round] (0.25,0.75) -- (0.25,1.25); 
 \draw [line width=3pt, line cap=round] (0.75,0.25) -- (0.75,0.75); 
 \draw [line width=3pt, line cap=round] (0.25,0.75) -- (0.25,0.25); 
 \draw [line width=3pt, line cap=round] (0.25,0.25) -- (0.75,0.25); 
 \draw [line width=3pt, line cap=round] (1,1.5) -- (0.75,1.25); 
 \draw [line width=3pt, line cap=round] (0.75,1.75) -- (0.25,1.75); 
 \draw [line width=3pt, line cap=round] (0.75,1.75) -- (0.875,1.625); 
 \draw [line width=3pt, line cap=round] (0.25,1.75) -- (0.25,1.25); 
 \draw [line width=3pt, line cap=round] (1,1.5) -- (0.75,1.25); 
 \draw [line width=3pt, line cap=round] (0.75,1.25) -- (0.75,0.75); 
 \draw [line width=3pt, line cap=round] (1,1.5) -- (1.25,1.75); 
 \draw [line width=3pt, line cap=round] (1.75,0.25) -- (1.75,0.75); 
 \draw [line width=3pt, line cap=round] (1.75,0.75) -- (1.25,0.75); 
 \draw [line width=3pt, line cap=round] (1.25,0.75) -- (1.25,0.25); 
 \draw [line width=3pt, line cap=round] (1.25,0.25) -- (1.75,0.25); 
 \draw [line width=3pt, line cap=round] (1.75,1.25) -- (1.75,1.75); 
 \draw [line width=3pt, line cap=round] (1.75,1.75) -- (1.25,1.75); 
 \draw [line width=3pt, line cap=round] (1,1.5) -- (1.25,1.75); 
 \draw [line width=3pt, line cap=round] (1.25,1.25) -- (1.75,1.25); 
 \draw [line width=3pt, line cap=round] (1.25,1.25) -- (1.125,1.375); 
\end{tikzpicture} 

\begin{tikzpicture}[framed,background rectangle/.style={ultra thick,draw=black}]
 \draw [line width=3pt, line cap=round] (0.5,1) -- (0.25,1.25); 
 \draw [line width=3pt, line cap=round] (0.75,0.25) -- (0.75,0.75); 
 \draw [line width=3pt, line cap=round] (0.5,1) -- (0.75,0.75); 
 \draw [line width=3pt, line cap=round] (0.25,0.75) -- (0.25,0.25); 
 \draw [line width=3pt, line cap=round] (0.25,0.75) -- (0.375,0.875); 
 \draw [line width=3pt, line cap=round] (0.25,0.25) -- (0.75,0.25); 
 \draw [line width=3pt, line cap=round] (0.75,1.75) -- (0.25,1.75); 
 \draw [line width=3pt, line cap=round] (0.25,1.75) -- (0.25,1.25); 
 \draw [line width=3pt, line cap=round] (0.5,1) -- (0.25,1.25); 
 \draw [line width=3pt, line cap=round] (1.25,1.25) -- (0.75,1.25); 
 \draw [line width=3pt, line cap=round] (0.75,1.25) -- (0.625,1.125); 
 \draw [line width=3pt, line cap=round] (0.5,1) -- (0.75,0.75); 
 \draw [line width=3pt, line cap=round] (0.75,1.75) -- (1.25,1.75); 
 \draw [line width=3pt, line cap=round] (1.75,0.25) -- (1.75,0.75); 
 \draw [line width=3pt, line cap=round] (1.75,0.75) -- (1.25,0.75); 
 \draw [line width=3pt, line cap=round] (1.25,0.75) -- (1.25,0.25); 
 \draw [line width=3pt, line cap=round] (1.25,0.25) -- (1.75,0.25); 
 \draw [line width=3pt, line cap=round] (1.75,1.25) -- (1.75,1.75); 
 \draw [line width=3pt, line cap=round] (1.75,1.75) -- (1.25,1.75); 
 \draw [line width=3pt, line cap=round] (1.25,1.25) -- (1.75,1.25); 
\end{tikzpicture} 

\begin{tikzpicture}[framed,background rectangle/.style={ultra thick,draw=black}]
 \draw [line width=3pt, line cap=round] (0.75,0.25) -- (0.75,0.75); 
 \draw [line width=3pt, line cap=round] (0.75,0.75) -- (0.25,0.75); 
 \draw [line width=3pt, line cap=round] (0.25,0.75) -- (0.25,0.25); 
 \draw [line width=3pt, line cap=round] (0.25,0.25) -- (0.75,0.25); 
 \draw [line width=3pt, line cap=round] (0.75,1.75) -- (0.25,1.75); 
 \draw [line width=3pt, line cap=round] (0.25,1.75) -- (0.25,1.25); 
 \draw [line width=3pt, line cap=round] (0.25,1.25) -- (0.75,1.25); 
 \draw [line width=3pt, line cap=round] (1.25,1.25) -- (0.75,1.25); 
 \draw [line width=3pt, line cap=round] (0.75,1.75) -- (1.25,1.75); 
 \draw [line width=3pt, line cap=round] (1.75,0.25) -- (1.75,0.75); 
 \draw [line width=3pt, line cap=round] (1.75,0.75) -- (1.25,0.75); 
 \draw [line width=3pt, line cap=round] (1.25,0.75) -- (1.25,0.25); 
 \draw [line width=3pt, line cap=round] (1.25,0.25) -- (1.75,0.25); 
 \draw [line width=3pt, line cap=round] (1.75,1.25) -- (1.75,1.75); 
 \draw [line width=3pt, line cap=round] (1.75,1.75) -- (1.25,1.75); 
 \draw [line width=3pt, line cap=round] (1.25,1.25) -- (1.75,1.25); 
\end{tikzpicture} 

\begin{tikzpicture}[framed,background rectangle/.style={ultra thick,draw=black}]
 \draw [line width=3pt, line cap=round] (0.25,0.75) -- (0.25,1.25); 
 \draw [line width=3pt, line cap=round] (0.75,0.25) -- (0.75,0.75); 
 \draw [line width=3pt, line cap=round] (0.25,0.75) -- (0.25,0.25); 
 \draw [line width=3pt, line cap=round] (0.25,0.25) -- (0.75,0.25); 
 \draw [line width=3pt, line cap=round] (0.75,1.75) -- (0.25,1.75); 
 \draw [line width=3pt, line cap=round] (0.25,1.75) -- (0.25,1.25); 
 \draw [line width=3pt, line cap=round] (1.25,1.25) -- (0.75,1.25); 
 \draw [line width=3pt, line cap=round] (0.75,1.25) -- (0.75,0.75); 
 \draw [line width=3pt, line cap=round] (0.75,1.75) -- (1.25,1.75); 
 \draw [line width=3pt, line cap=round] (1.75,0.25) -- (1.75,0.75); 
 \draw [line width=3pt, line cap=round] (1.75,0.75) -- (1.25,0.75); 
 \draw [line width=3pt, line cap=round] (1.25,0.75) -- (1.25,0.25); 
 \draw [line width=3pt, line cap=round] (1.25,0.25) -- (1.75,0.25); 
 \draw [line width=3pt, line cap=round] (1.75,1.25) -- (1.75,1.75); 
 \draw [line width=3pt, line cap=round] (1.75,1.75) -- (1.25,1.75); 
 \draw [line width=3pt, line cap=round] (1.25,1.25) -- (1.75,1.25); 
\end{tikzpicture} 

\begin{tikzpicture}[framed,background rectangle/.style={ultra thick,draw=black}]
 \draw [line width=3pt, line cap=round] (0.5,1) -- (0.25,1.25); 
 \draw [line width=3pt, line cap=round] (0.75,0.25) -- (0.75,0.75); 
 \draw [line width=3pt, line cap=round] (0.5,1) -- (0.75,0.75); 
 \draw [line width=3pt, line cap=round] (0.25,0.75) -- (0.25,0.25); 
 \draw [line width=3pt, line cap=round] (0.25,0.75) -- (0.375,0.875); 
 \draw [line width=3pt, line cap=round] (0.25,0.25) -- (0.75,0.25); 
 \draw [line width=3pt, line cap=round] (0.75,1.25) -- (0.75,1.75); 
 \draw [line width=3pt, line cap=round] (0.75,1.25) -- (0.625,1.125); 
 \draw [line width=3pt, line cap=round] (0.75,1.75) -- (0.25,1.75); 
 \draw [line width=3pt, line cap=round] (0.25,1.75) -- (0.25,1.25); 
 \draw [line width=3pt, line cap=round] (0.5,1) -- (0.25,1.25); 
 \draw [line width=3pt, line cap=round] (0.5,1) -- (0.75,0.75); 
 \draw [line width=3pt, line cap=round] (1.75,0.25) -- (1.75,0.75); 
 \draw [line width=3pt, line cap=round] (1.75,0.75) -- (1.25,0.75); 
 \draw [line width=3pt, line cap=round] (1.25,0.75) -- (1.25,0.25); 
 \draw [line width=3pt, line cap=round] (1.25,0.25) -- (1.75,0.25); 
 \draw [line width=3pt, line cap=round] (1.75,1.25) -- (1.75,1.75); 
 \draw [line width=3pt, line cap=round] (1.75,1.75) -- (1.25,1.75); 
 \draw [line width=3pt, line cap=round] (1.25,1.75) -- (1.25,1.25); 
 \draw [line width=3pt, line cap=round] (1.25,1.25) -- (1.75,1.25); 
\end{tikzpicture} 


\noindent
\begin{tikzpicture}[framed,background rectangle/.style={ultra thick,draw=black}]
 \draw [line width=3pt, line cap=round] (0.75,0.25) -- (0.75,0.75); 
 \draw [line width=3pt, line cap=round] (0.75,0.75) -- (0.25,0.75); 
 \draw [line width=3pt, line cap=round] (0.25,0.75) -- (0.25,0.25); 
 \draw [line width=3pt, line cap=round] (0.25,0.25) -- (0.75,0.25); 
 \draw [line width=3pt, line cap=round] (0.75,1.25) -- (0.75,1.75); 
 \draw [line width=3pt, line cap=round] (0.75,1.75) -- (0.25,1.75); 
 \draw [line width=3pt, line cap=round] (0.25,1.75) -- (0.25,1.25); 
 \draw [line width=3pt, line cap=round] (0.25,1.25) -- (0.75,1.25); 
 \draw [line width=3pt, line cap=round] (1.75,0.25) -- (1.75,0.75); 
 \draw [line width=3pt, line cap=round] (1.75,0.75) -- (1.25,0.75); 
 \draw [line width=3pt, line cap=round] (1.25,0.75) -- (1.25,0.25); 
 \draw [line width=3pt, line cap=round] (1.25,0.25) -- (1.75,0.25); 
 \draw [line width=3pt, line cap=round] (1.75,1.25) -- (1.75,1.75); 
 \draw [line width=3pt, line cap=round] (1.75,1.75) -- (1.25,1.75); 
 \draw [line width=3pt, line cap=round] (1.25,1.75) -- (1.25,1.25); 
 \draw [line width=3pt, line cap=round] (1.25,1.25) -- (1.75,1.25); 
\end{tikzpicture} 

\begin{tikzpicture}[framed,background rectangle/.style={ultra thick,draw=black}]
 \draw [line width=3pt, line cap=round] (0.25,0.75) -- (0.25,1.25); 
 \draw [line width=3pt, line cap=round] (0.75,0.25) -- (0.75,0.75); 
 \draw [line width=3pt, line cap=round] (0.25,0.75) -- (0.25,0.25); 
 \draw [line width=3pt, line cap=round] (0.25,0.25) -- (0.75,0.25); 
 \draw [line width=3pt, line cap=round] (0.75,1.25) -- (0.75,1.75); 
 \draw [line width=3pt, line cap=round] (0.75,1.75) -- (0.25,1.75); 
 \draw [line width=3pt, line cap=round] (0.25,1.75) -- (0.25,1.25); 
 \draw [line width=3pt, line cap=round] (0.75,1.25) -- (0.75,0.75); 
 \draw [line width=3pt, line cap=round] (1.75,0.25) -- (1.75,0.75); 
 \draw [line width=3pt, line cap=round] (1.75,0.75) -- (1.25,0.75); 
 \draw [line width=3pt, line cap=round] (1.25,0.75) -- (1.25,0.25); 
 \draw [line width=3pt, line cap=round] (1.25,0.25) -- (1.75,0.25); 
 \draw [line width=3pt, line cap=round] (1.75,1.25) -- (1.75,1.75); 
 \draw [line width=3pt, line cap=round] (1.75,1.75) -- (1.25,1.75); 
 \draw [line width=3pt, line cap=round] (1.25,1.75) -- (1.25,1.25); 
 \draw [line width=3pt, line cap=round] (1.25,1.25) -- (1.75,1.25); 
\end{tikzpicture} 

\begin{tikzpicture}[framed,background rectangle/.style={ultra thick,draw=black}]
 \draw [line width=3pt, line cap=round] (0.5,1) -- (0.25,1.25); 
 \draw [line width=3pt, line cap=round] (0.75,0.25) -- (0.75,0.75); 
 \draw [line width=3pt, line cap=round] (0.5,1) -- (0.75,0.75); 
 \draw [line width=3pt, line cap=round] (0.25,0.75) -- (0.25,0.25); 
 \draw [line width=3pt, line cap=round] (0.25,0.75) -- (0.375,0.875); 
 \draw [line width=3pt, line cap=round] (0.25,0.25) -- (0.75,0.25); 
 \draw [line width=3pt, line cap=round] (1,1.5) -- (0.75,1.25); 
 \draw [line width=3pt, line cap=round] (0.75,1.25) -- (0.625,1.125); 
 \draw [line width=3pt, line cap=round] (0.75,1.75) -- (0.25,1.75); 
 \draw [line width=3pt, line cap=round] (0.75,1.75) -- (0.875,1.625); 
 \draw [line width=3pt, line cap=round] (0.25,1.75) -- (0.25,1.25); 
 \draw [line width=3pt, line cap=round] (0.5,1) -- (0.25,1.25); 
 \draw [line width=3pt, line cap=round] (1.25,0.75) -- (1.25,1.25); 
 \draw [line width=3pt, line cap=round] (1.25,1.25) -- (1.125,1.375); 
 \draw [line width=3pt, line cap=round] (1,1.5) -- (0.75,1.25); 
 \draw [line width=3pt, line cap=round] (0.5,1) -- (0.75,0.75); 
 \draw [line width=3pt, line cap=round] (1,1.5) -- (1.25,1.75); 
 \draw [line width=3pt, line cap=round] (1.75,0.25) -- (1.75,0.75); 
 \draw [line width=3pt, line cap=round] (1.25,0.75) -- (1.25,0.25); 
 \draw [line width=3pt, line cap=round] (1.25,0.25) -- (1.75,0.25); 
 \draw [line width=3pt, line cap=round] (1.75,1.25) -- (1.75,1.75); 
 \draw [line width=3pt, line cap=round] (1.75,1.75) -- (1.25,1.75); 
 \draw [line width=3pt, line cap=round] (1,1.5) -- (1.25,1.75); 
 \draw [line width=3pt, line cap=round] (1.75,1.25) -- (1.75,0.75); 
\end{tikzpicture} 

\begin{tikzpicture}[framed,background rectangle/.style={ultra thick,draw=black}]
 \draw [line width=3pt, line cap=round] (0.75,0.25) -- (0.75,0.75); 
 \draw [line width=3pt, line cap=round] (0.75,0.75) -- (0.25,0.75); 
 \draw [line width=3pt, line cap=round] (0.25,0.75) -- (0.25,0.25); 
 \draw [line width=3pt, line cap=round] (0.25,0.25) -- (0.75,0.25); 
 \draw [line width=3pt, line cap=round] (1,1.5) -- (0.75,1.25); 
 \draw [line width=3pt, line cap=round] (0.75,1.75) -- (0.25,1.75); 
 \draw [line width=3pt, line cap=round] (0.75,1.75) -- (0.875,1.625); 
 \draw [line width=3pt, line cap=round] (0.25,1.75) -- (0.25,1.25); 
 \draw [line width=3pt, line cap=round] (0.25,1.25) -- (0.75,1.25); 
 \draw [line width=3pt, line cap=round] (1.25,0.75) -- (1.25,1.25); 
 \draw [line width=3pt, line cap=round] (1.25,1.25) -- (1.125,1.375); 
 \draw [line width=3pt, line cap=round] (1,1.5) -- (0.75,1.25); 
 \draw [line width=3pt, line cap=round] (1,1.5) -- (1.25,1.75); 
 \draw [line width=3pt, line cap=round] (1.75,0.25) -- (1.75,0.75); 
 \draw [line width=3pt, line cap=round] (1.25,0.75) -- (1.25,0.25); 
 \draw [line width=3pt, line cap=round] (1.25,0.25) -- (1.75,0.25); 
 \draw [line width=3pt, line cap=round] (1.75,1.25) -- (1.75,1.75); 
 \draw [line width=3pt, line cap=round] (1.75,1.75) -- (1.25,1.75); 
 \draw [line width=3pt, line cap=round] (1,1.5) -- (1.25,1.75); 
 \draw [line width=3pt, line cap=round] (1.75,1.25) -- (1.75,0.75); 
\end{tikzpicture} 

\begin{tikzpicture}[framed,background rectangle/.style={ultra thick,draw=black}]
 \draw [line width=3pt, line cap=round] (0.25,0.75) -- (0.25,1.25); 
 \draw [line width=3pt, line cap=round] (0.75,0.25) -- (0.75,0.75); 
 \draw [line width=3pt, line cap=round] (0.25,0.75) -- (0.25,0.25); 
 \draw [line width=3pt, line cap=round] (0.25,0.25) -- (0.75,0.25); 
 \draw [line width=3pt, line cap=round] (1,1.5) -- (0.75,1.25); 
 \draw [line width=3pt, line cap=round] (0.75,1.75) -- (0.25,1.75); 
 \draw [line width=3pt, line cap=round] (0.75,1.75) -- (0.875,1.625); 
 \draw [line width=3pt, line cap=round] (0.25,1.75) -- (0.25,1.25); 
 \draw [line width=3pt, line cap=round] (1.25,0.75) -- (1.25,1.25); 
 \draw [line width=3pt, line cap=round] (1.25,1.25) -- (1.125,1.375); 
 \draw [line width=3pt, line cap=round] (1,1.5) -- (0.75,1.25); 
 \draw [line width=3pt, line cap=round] (0.75,1.25) -- (0.75,0.75); 
 \draw [line width=3pt, line cap=round] (1,1.5) -- (1.25,1.75); 
 \draw [line width=3pt, line cap=round] (1.75,0.25) -- (1.75,0.75); 
 \draw [line width=3pt, line cap=round] (1.25,0.75) -- (1.25,0.25); 
 \draw [line width=3pt, line cap=round] (1.25,0.25) -- (1.75,0.25); 
 \draw [line width=3pt, line cap=round] (1.75,1.25) -- (1.75,1.75); 
 \draw [line width=3pt, line cap=round] (1.75,1.75) -- (1.25,1.75); 
 \draw [line width=3pt, line cap=round] (1,1.5) -- (1.25,1.75); 
 \draw [line width=3pt, line cap=round] (1.75,1.25) -- (1.75,0.75); 
\end{tikzpicture} 


\noindent
\begin{tikzpicture}[framed,background rectangle/.style={ultra thick,draw=black}]
 \draw [line width=3pt, line cap=round] (0.5,1) -- (0.25,1.25); 
 \draw [line width=3pt, line cap=round] (0.75,0.25) -- (0.75,0.75); 
 \draw [line width=3pt, line cap=round] (0.5,1) -- (0.75,0.75); 
 \draw [line width=3pt, line cap=round] (0.25,0.75) -- (0.25,0.25); 
 \draw [line width=3pt, line cap=round] (0.25,0.75) -- (0.375,0.875); 
 \draw [line width=3pt, line cap=round] (0.25,0.25) -- (0.75,0.25); 
 \draw [line width=3pt, line cap=round] (0.75,1.75) -- (0.25,1.75); 
 \draw [line width=3pt, line cap=round] (0.25,1.75) -- (0.25,1.25); 
 \draw [line width=3pt, line cap=round] (0.5,1) -- (0.25,1.25); 
 \draw [line width=3pt, line cap=round] (1.25,0.75) -- (1.25,1.25); 
 \draw [line width=3pt, line cap=round] (1.25,1.25) -- (0.75,1.25); 
 \draw [line width=3pt, line cap=round] (0.75,1.25) -- (0.625,1.125); 
 \draw [line width=3pt, line cap=round] (0.5,1) -- (0.75,0.75); 
 \draw [line width=3pt, line cap=round] (0.75,1.75) -- (1.25,1.75); 
 \draw [line width=3pt, line cap=round] (1.75,0.25) -- (1.75,0.75); 
 \draw [line width=3pt, line cap=round] (1.25,0.75) -- (1.25,0.25); 
 \draw [line width=3pt, line cap=round] (1.25,0.25) -- (1.75,0.25); 
 \draw [line width=3pt, line cap=round] (1.75,1.25) -- (1.75,1.75); 
 \draw [line width=3pt, line cap=round] (1.75,1.75) -- (1.25,1.75); 
 \draw [line width=3pt, line cap=round] (1.75,1.25) -- (1.75,0.75); 
\end{tikzpicture} 

\begin{tikzpicture}[framed,background rectangle/.style={ultra thick,draw=black}]
 \draw [line width=3pt, line cap=round] (0.75,0.25) -- (0.75,0.75); 
 \draw [line width=3pt, line cap=round] (0.75,0.75) -- (0.25,0.75); 
 \draw [line width=3pt, line cap=round] (0.25,0.75) -- (0.25,0.25); 
 \draw [line width=3pt, line cap=round] (0.25,0.25) -- (0.75,0.25); 
 \draw [line width=3pt, line cap=round] (0.75,1.75) -- (0.25,1.75); 
 \draw [line width=3pt, line cap=round] (0.25,1.75) -- (0.25,1.25); 
 \draw [line width=3pt, line cap=round] (0.25,1.25) -- (0.75,1.25); 
 \draw [line width=3pt, line cap=round] (1.25,0.75) -- (1.25,1.25); 
 \draw [line width=3pt, line cap=round] (1.25,1.25) -- (0.75,1.25); 
 \draw [line width=3pt, line cap=round] (0.75,1.75) -- (1.25,1.75); 
 \draw [line width=3pt, line cap=round] (1.75,0.25) -- (1.75,0.75); 
 \draw [line width=3pt, line cap=round] (1.25,0.75) -- (1.25,0.25); 
 \draw [line width=3pt, line cap=round] (1.25,0.25) -- (1.75,0.25); 
 \draw [line width=3pt, line cap=round] (1.75,1.25) -- (1.75,1.75); 
 \draw [line width=3pt, line cap=round] (1.75,1.75) -- (1.25,1.75); 
 \draw [line width=3pt, line cap=round] (1.75,1.25) -- (1.75,0.75); 
\end{tikzpicture} 

\begin{tikzpicture}[framed,background rectangle/.style={ultra thick,draw=black}]
 \draw [line width=3pt, line cap=round] (0.25,0.75) -- (0.25,1.25); 
 \draw [line width=3pt, line cap=round] (0.75,0.25) -- (0.75,0.75); 
 \draw [line width=3pt, line cap=round] (0.25,0.75) -- (0.25,0.25); 
 \draw [line width=3pt, line cap=round] (0.25,0.25) -- (0.75,0.25); 
 \draw [line width=3pt, line cap=round] (0.75,1.75) -- (0.25,1.75); 
 \draw [line width=3pt, line cap=round] (0.25,1.75) -- (0.25,1.25); 
 \draw [line width=3pt, line cap=round] (1.25,0.75) -- (1.25,1.25); 
 \draw [line width=3pt, line cap=round] (1.25,1.25) -- (0.75,1.25); 
 \draw [line width=3pt, line cap=round] (0.75,1.25) -- (0.75,0.75); 
 \draw [line width=3pt, line cap=round] (0.75,1.75) -- (1.25,1.75); 
 \draw [line width=3pt, line cap=round] (1.75,0.25) -- (1.75,0.75); 
 \draw [line width=3pt, line cap=round] (1.25,0.75) -- (1.25,0.25); 
 \draw [line width=3pt, line cap=round] (1.25,0.25) -- (1.75,0.25); 
 \draw [line width=3pt, line cap=round] (1.75,1.25) -- (1.75,1.75); 
 \draw [line width=3pt, line cap=round] (1.75,1.75) -- (1.25,1.75); 
 \draw [line width=3pt, line cap=round] (1.75,1.25) -- (1.75,0.75); 
\end{tikzpicture} 

\begin{tikzpicture}[framed,background rectangle/.style={ultra thick,draw=black}]
 \draw [line width=3pt, line cap=round] (0.5,1) -- (0.25,1.25); 
 \draw [line width=3pt, line cap=round] (0.75,0.25) -- (0.75,0.75); 
 \draw [line width=3pt, line cap=round] (0.5,1) -- (0.75,0.75); 
 \draw [line width=3pt, line cap=round] (0.25,0.75) -- (0.25,0.25); 
 \draw [line width=3pt, line cap=round] (0.25,0.75) -- (0.375,0.875); 
 \draw [line width=3pt, line cap=round] (0.25,0.25) -- (0.75,0.25); 
 \draw [line width=3pt, line cap=round] (0.75,1.25) -- (0.75,1.75); 
 \draw [line width=3pt, line cap=round] (0.75,1.25) -- (0.625,1.125); 
 \draw [line width=3pt, line cap=round] (0.75,1.75) -- (0.25,1.75); 
 \draw [line width=3pt, line cap=round] (0.25,1.75) -- (0.25,1.25); 
 \draw [line width=3pt, line cap=round] (0.5,1) -- (0.25,1.25); 
 \draw [line width=3pt, line cap=round] (1.25,0.75) -- (1.25,1.25); 
 \draw [line width=3pt, line cap=round] (0.5,1) -- (0.75,0.75); 
 \draw [line width=3pt, line cap=round] (1.75,0.25) -- (1.75,0.75); 
 \draw [line width=3pt, line cap=round] (1.25,0.75) -- (1.25,0.25); 
 \draw [line width=3pt, line cap=round] (1.25,0.25) -- (1.75,0.25); 
 \draw [line width=3pt, line cap=round] (1.75,1.25) -- (1.75,1.75); 
 \draw [line width=3pt, line cap=round] (1.75,1.75) -- (1.25,1.75); 
 \draw [line width=3pt, line cap=round] (1.25,1.75) -- (1.25,1.25); 
 \draw [line width=3pt, line cap=round] (1.75,1.25) -- (1.75,0.75); 
\end{tikzpicture} 

\begin{tikzpicture}[framed,background rectangle/.style={ultra thick,draw=black}]
 \draw [line width=3pt, line cap=round] (0.75,0.25) -- (0.75,0.75); 
 \draw [line width=3pt, line cap=round] (0.75,0.75) -- (0.25,0.75); 
 \draw [line width=3pt, line cap=round] (0.25,0.75) -- (0.25,0.25); 
 \draw [line width=3pt, line cap=round] (0.25,0.25) -- (0.75,0.25); 
 \draw [line width=3pt, line cap=round] (0.75,1.25) -- (0.75,1.75); 
 \draw [line width=3pt, line cap=round] (0.75,1.75) -- (0.25,1.75); 
 \draw [line width=3pt, line cap=round] (0.25,1.75) -- (0.25,1.25); 
 \draw [line width=3pt, line cap=round] (0.25,1.25) -- (0.75,1.25); 
 \draw [line width=3pt, line cap=round] (1.25,0.75) -- (1.25,1.25); 
 \draw [line width=3pt, line cap=round] (1.75,0.25) -- (1.75,0.75); 
 \draw [line width=3pt, line cap=round] (1.25,0.75) -- (1.25,0.25); 
 \draw [line width=3pt, line cap=round] (1.25,0.25) -- (1.75,0.25); 
 \draw [line width=3pt, line cap=round] (1.75,1.25) -- (1.75,1.75); 
 \draw [line width=3pt, line cap=round] (1.75,1.75) -- (1.25,1.75); 
 \draw [line width=3pt, line cap=round] (1.25,1.75) -- (1.25,1.25); 
 \draw [line width=3pt, line cap=round] (1.75,1.25) -- (1.75,0.75); 
\end{tikzpicture} 


\noindent
\begin{tikzpicture}[framed,background rectangle/.style={ultra thick,draw=black}]
 \draw [line width=3pt, line cap=round] (0.25,0.75) -- (0.25,1.25); 
 \draw [line width=3pt, line cap=round] (0.75,0.25) -- (0.75,0.75); 
 \draw [line width=3pt, line cap=round] (0.25,0.75) -- (0.25,0.25); 
 \draw [line width=3pt, line cap=round] (0.25,0.25) -- (0.75,0.25); 
 \draw [line width=3pt, line cap=round] (0.75,1.25) -- (0.75,1.75); 
 \draw [line width=3pt, line cap=round] (0.75,1.75) -- (0.25,1.75); 
 \draw [line width=3pt, line cap=round] (0.25,1.75) -- (0.25,1.25); 
 \draw [line width=3pt, line cap=round] (1.25,0.75) -- (1.25,1.25); 
 \draw [line width=3pt, line cap=round] (0.75,1.25) -- (0.75,0.75); 
 \draw [line width=3pt, line cap=round] (1.75,0.25) -- (1.75,0.75); 
 \draw [line width=3pt, line cap=round] (1.25,0.75) -- (1.25,0.25); 
 \draw [line width=3pt, line cap=round] (1.25,0.25) -- (1.75,0.25); 
 \draw [line width=3pt, line cap=round] (1.75,1.25) -- (1.75,1.75); 
 \draw [line width=3pt, line cap=round] (1.75,1.75) -- (1.25,1.75); 
 \draw [line width=3pt, line cap=round] (1.25,1.75) -- (1.25,1.25); 
 \draw [line width=3pt, line cap=round] (1.75,1.25) -- (1.75,0.75); 
\end{tikzpicture} 



\end{minipage}
}
